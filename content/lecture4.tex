%####%&&&&%####%&&&&%####%&&&&%####%&&&&%####%&&&&%####%&&&&%####%&&&&%####%&&&&

\section{Lecture 4: Marine Sediments}

\subsection{Marine sediments}

\begin{enumerate}
    \item Sediment layers represent a record of Earth history, including:
    \begin{itemize}
        \item Movement of tectonic plates
        \item Past changes in climate
        \item Ancient ocean circulation patterns
        \item Cataclysmic events (meteorites, tsunamis, volcanic eruptions)
    \end{itemize}
    \item Are a critical and active component of global
        biochemical cycles
    \item Marine sediments contain valuable economic resources
    \item Contain an enormous amount of biomass -- \textit{A frontier
        area of microbiological life} and a critical component of the global
        carbon cycle
\end{enumerate}

\subsection{Sampling the seafloor}

\begin{itemize}
    \item Coring
    \item Coring tools are simple in design: a long metal pipe, with a
        very heavy weight 100-200kg, that falls down until it hits the
        seafloor
    \item Other tools for sampling the seafloor
\end{itemize}

In a neritic deposit you will recover 10ka of Earth's history
from 10m long sediment core.

In a Pelagic deposit, you can capture over 1Ma of Earth's history
from 10m long sediment core.

\subsection{Scientific drilling vs seabed coring}

\begin{itemize}
    \item A pipe is lowered to the bottom of the sea
    \item Inside of the pipe is a wire with a drill at the end
\end{itemize}

\subsection{Four main types of sediment}

\begin{itemize}
    \item \textbf{Lithogenous (terrigenous)} -- composed of fragments of
        pre-existing rock material
    \item \textbf{Biogenous} -- composed of hard remains of once-living
        organisms
    \item \textbf{Hydrogenous} -- formed when dissolved materials come out of
        solution (precipitate)
    \item \textbf{Cosmogenous} -- derived from outer space
\end{itemize}

\subsection{Lithogenous sediment texture}

\begin{itemize}
    \item High energy $\longrightarrow$ low energy
    \item Bach to shallow marine, transition to deep marine
\end{itemize}

\subsection{Water-river transport and run-off}

In addition to sediment, the dissolved load is also an important contributor
to deep-sea sedimentation. It contains PO$_4$, NO$_3-$, and other nutrients
needed for plant growth, such as Ca+, HCO$_3$-, and H$_4$SiO$_4$, from which
pelagic organisms build their shells and skeletons.

\subsection{River transport and run-off}

Operates across the globe, from equatorial to polar regions.

\subsection{Windblown sediment}

Eolian (wind blown) sediment can be traced by looking at the quartz
concentration in some marine sediments.

Accumulation rates of windblown sediments in the deep sea are typically up to
a few mm/1000 years.

\subsection{Ice -- glaciers (icebergs) and sea ice}

They can carry coarse-grained material very far out to the sea.

\subsection{Turbidity currents and mass transport deposits}

\textbf{Graded bedding}: As the turbidity current slows, larger grains
settle first, followed by progressively finer grains.

\subsection{Biogenous: photosynthetic phytoplankton}

Phytoplankton blooms in the Barents Sea, off the northern coast of Norway

\subsection{Origin and composition of biogenous sediment}

\begin{itemize}
    \item Organisms that produce hard parts die
    \item Material rains down on the ocean floor and accumulates as:
        \begin{itemize}
            \item Microscopic tests (shells)
            \item If sediment is made of at least 30\% shell material, it is
                called biogenous ooze
        \end{itemize}
    \item Microscopic biogenous shells are composed of 2 main chemical
        compounds:
        \begin{itemize}
            \item Silica (SiO$_2$): Diatoms (algae -- plants), Radiolarian
                (protozoan -- animals)
            \item Calcium carbonate or calcite (CaCO$_3$):
                Coccolithophores (algae -- plants), Foraminifers
                (protozoan -- animals)
        \end{itemize}
\end{itemize}

\subsection{Lithified biogenous sediment}

When biogenous ooze hardens and lithifies, it can form:
Diatomaceous earth (if composed of diatom-rich ooze) or Chalk (if composed of
coccolith-rich ooze).

\subsection{Distribution of Biogenous sediment}

\begin{itemize}
    \item Most biogenous ooze is found as pelagic (open ocean) deposits
    \item Factors affecting the distribution of biogenous ooze:
        \begin{itemize}
            \item Productivity (amount of organisms living in surface waters)
            \item Destruction (dissolving at depth)
            \item Dilution (mixing with lithogenous clays)
        \end{itemize}
    \item Productivity is usually nutrient limited and is overcome in
        (1) areas where lithogenous sediment is delivered and (2) regions of
        upwelling
\end{itemize}

\subsection{Distribution of siliceous ooze}

Silica is generally undersaturated in seawater, so silicate shells steadily
dissolve in the water column and at the seafloor.

\subsection{Distribution of carbonate ooze}

Calcite compensation depth (CCD) is the depth in the oceans below which the
rate of supply of calcite (calcium carbonate) lags behind the rate of
dissolution in the water column.

Calcite dissolves beneath the calcite compensation depth (CCD) at about 4.5km
average depth.

The CCD is influenced by pressure, temperature, amount of CO$_2$ dissolved in
sewater, and pH water. Therefore it varies in different parts of the ocean.

\subsection{Distribution of carbonate sediment}

Mostly around shallow seafloor -- around mid-ocean ridges.

\subsection{Pelagic clays -- 'red clays'}

Slowly accumulating 'red clays' are found in the large ocean gyres, where
little productivity occurs, and where sediments are mainly fine grained
terrigenous material carried by wind.

%####%&&&&%####%&&&&%####%&&&&%####%&&&&%####%&&&&%####%&&&&%####%&&&&%####%&&&&