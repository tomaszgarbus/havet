\section{Chapter 6: Air-Sea Interaction}

\subsection{Glossary of terms}
\begin{itemize}
	\item \textbf{plane of the ecliptic} -- the surface connecting all
		points of Earth's orbit
	\item \textbf{vernal equinox} -- the passage of Sun across the equator
		as it moves from the Southern Hemisphere into the Northern
		Hemisphere, approximately March 21. During this time, all
		places in the world experience equal lengths of night and day.
		Also known as the spring equinox.
	\item \textbf{summer solstice} -- in the Northern Hemishpere, it is
		the instant when the Sun moves north to the Tropic of Cancer
		before changing direction and moving southward toward the
		equator approximately June 21.
	\item \textbf{Tropic of Cancer} -- the latitude 23.5 degrees north,
		which is the furthest location north that receives vertical
		rays of the sun.
	\item \textbf{autumnal equinox} -- the passage of Sun across the
		equator as it moves from the Northern Hemisphere, approximately
		September 23. During this time, all places in the world
		experience equal lengths of night and day. Also called fall
		equinox.
	\item \textbf{winter solstice} -- the instant the southward-moving Sun
		reaches the Tropic of Cancer before changing firection and
		moving north back toward the equator, approximately December
		21.
	\item \textbf{Tropic of Capricorn} -- the latitude 23.5 degrees south
		which is the furthest location south that receives vertical
		rays of the Sun.
	\item \textbf{declination} -- the angular distance of the Sun or Moon
		above or below the plane of Earth's equator
	\item \textbf{tropics} -- the region of Earth's surface lying between
		the Tropic of Cancer and the Tropic of Capricorn. Also known as
		the Torrid Zone.
	\item \textbf{Arctic Circle} -- the latitude 66.5 degrees north
	\item \textbf{Antarctic Circle} -- the latitude 66.5 degrees south
	\item \textbf{albedo} -- the fraction of incident electromagnetic
		radiation reflected by a surface
	\item \textbf{troposphere} -- the lowermost portion of the atmosphere
		which extends from Earth's surface to 12 kilometer. It is where
		all the weather is produced.
	\item \textbf{convection cell} -- a circular-moving loop of matter
		involved in convective movement
	\item \textbf{wind} -- the movement of air, usually as a result of
		pressure differences
	\item \textbf{Coriolis effect} -- an apparent force resulting from
		Earth's rotation that causes particles in motion to be
		deflected to the right in the Northern Hemisphere and to the
		eft in the Southern Hemisphere
	\item \textbf{Hadley cells} -- the large atmospheric circulation cell
		that occurs between the equator and 30\% latitude in each
		hemisphere
	\item \textbf{Ferrel cell} -- the large atmospheric circulation cell
		that occurs between 30 and 60 degrees latitude in each
		hemisphere
	\item \textbf{polar cell} -- the large atmospheric circulation cell
		that occurs between 60 and 90 degrees latitude in each
		hemisphere
	\item \textbf{subtropical highs} -- a region of high atmospheric
		pressure located at about 30 degrees latitude
	\item \textbf{polar highs} -- the region of high atmospheric
		pressure that occurs at the poles in both hemispheres
	\item \textbf{equatorial flow} -- 
	\item \textbf{subpolar flow} --
	\item \textbf{trade winds} -- a global wind belt that moves from a
		subtropical high-pressure belt at about 30 deg north or south
		latitude toward the equatorial region. These winds move from a
		northeasterly direction in the NH and from a southeasterly
		direction in the SH.
	\item \textbf{prevailing westerly wind belts} -- a global wind belt
		that moves from a subtropical high-pressure belt at about
		30 degrees north or south latitude toward the polar front at
		about 60 degrees north or south latitude. These winds move from
		a southwesterly direction in the NH and from a northwesterly 
		direction in the SH.
	\item \textbf{polar easterly wind belts} --
	\item \textbf{doldrum} --
	\item \textbf{Intertropical Convergence Zone (ITCZ)} --
	\item \textbf{horse latitudes} --
	\item \textbf{polar front} --
	\item \textbf{weather} --
	\item \textbf{climate} --
	\item \textbf{cyclonic flow} --
	\item \textbf{anticyclonic flow} --
	\item \textbf{sea breezes} --
	\item \textbf{land breezez} --
	\item \textbf{storms} --
	\item \textbf{air masses} --
	\item \textbf{warm front} --
	\item \textbf{cold front} --
	\item \textbf{jet stream} --
	\item \textbf{tropical cyclone} --
	\item \textbf{hurricane} --
	\item \textbf{cyclone} --
	\item \textbf{typhoon} --
	\item \textbf{Saffir-Simpson scale} --
	\item \textbf{eye of the hurricane} --
	\item \textbf{storm surge} --
	\item \textbf{equatorial} --
	\item \textbf{tropical} --
	\item \textbf{subtropical} --
	\item \textbf{temperate} --
	\item \textbf{subpolar} --
	\item \textbf{sea ice} --
	\item \textbf{iceberg} --
	\item \textbf{pancake ice} --
	\item \textbf{ice floes} --
	\item \textbf{pressure ridges} --
	\item \textbf{shelf ice} --
\end{itemize}

\section{Chapter 7}
