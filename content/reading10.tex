\section{Chapter 12: Marine Life and the Marine Environment}

\subsection{Glossary of terms}
\begin{itemize}
	\item \textbf{bacteria} -- one of three major domains of life. The
		domain includes unicellural, prokaryotic microorganisms that
		vary in terms of morpholofy, oxygen and nutritional
		requirements, and motility.
	\item \textbf{archaea} -- one of three major domains of life. The
		domain consists of simple microscopic bacterialike creatures
		(including methane producers and sulfur oxidizers that inhabit
		deep-sea vents and seeps) and other microscopic life-forms that
		prefer environments of extreme conditions of temperature
		and/or pressure.
	\item \textbf{eukarya} -- one of three major domains of life. The
		domain includes single-celled or multicellural  organisms whose
		cells usually contain a distinct membrane-bound nucleus.
	\item \textbf{eubacteria} --
	\item \textbf{archaebacteria} --
	\item \textbf{plantae} --
	\item \textbf{animalia} --
	\item \textbf{fungi} --
	\item \textbf{protista} --
	\item \textbf{protozoa} --
	\item \textbf{species} --
	\item \textbf{plankton} --
	\item \textbf{plankter} --
	\item \textbf{biomass} --
	\item \textbf{autotrophic} -- algae, plants, and bacteria that can
		synthesize organic compounds from inorganic nutrients
	\item \textbf{phytoplankton} --
	\item \textbf{heterotrophic} --
	\item \textbf{zooplankton} --
	\item \textbf{bacterioplankton} --
	\item \textbf{virioplankton} --
	\item \textbf{holoplankton} --
	\item \textbf{meroplankton} --
	\item \textbf{macroplankton} --
	\item \textbf{picoplankton} --
	\item \textbf{benthos} -- the forms of marine life that live on the
		ocean bottom
	\item \textbf{epifauna} --
	\item \textbf{infauna} --
	\item \textbf{nektobenthos} --
	\item \textbf{pelagic sediment} -- sediment composed primarily of fine
		lithogenous and biogenous particles that is deposited slowly
		on the deep ocean floor: also called pelagic deposits
	\item \textbf{pelagic environment} -- the open-ocean environment which
		is divided into the neritic province (water depth 0 to 200
		meters or 656 feet) and the oceanic province (water depth
		greater than 200 meters or 656 feet)
	\item \textbf{protoplasm} -- the self-perpetuating living material
		making up all organisms, mostly consisting of the elements
		carbon, hydrogen, and oxygen combined into various chemical
		forms.
	\item \textbf{viscosity} -- a property of a substance to offer
		resistance to flow caused by internal friction.
	\item \textbf{streamlining} -- the shaping of an object so it produces
		the minimum of turbulence while moving through a fluid
		medium. The teardrop shape displays a high degree of
		streamlining.
	\item \textbf{broadcast spawning} --
	\item \textbf{stenothermal} -- pertaining to organisms that can
		withstand only a small change of temperature change
	\item \textbf{eurythermal} --
	\item \textbf{euryhaline} --
	\item \textbf{stenohaline} -- pertaining to organisms that can
		withstand only a small range of salinity change
	\item \textbf{osmosis} -- the process by which water molecules move
		through a semipermeable membrane from higher water molecule
		concentration (lower concentration) to lower water molecule
		concentration (higher salinity).
	\item \textbf{osmotic pressure} -- a measure of the tendency for
		osmosis to occur. It is the pressure that must be applied to
		the more concentrated solution to prevent the passage of water
		molecules into it from the less concentrated solution.
	\item \textbf{hypotonic} -- pertaining to the property of an aqueous
		solution having a lower osmotic pressure (salinity) than
		another aqueous solution, from which it is separated by a
		semipermeable membrane that will allow osmosis to occur. The
		hypotonic fluid will lose water molecules through the
		membrane to the other fluid.
	\item \textbf{gills} --
	\item \textbf{countershading} --
	\item \textbf{deep scattering layer (DSL)} -- a layer of marine
		organisms in the open ocean that scatter signals from an
		echo sounder. It migrates daily from depths of slightly over
		100 meters (330 feet) at night to more than 800 meters
		(2600 feet) during the day.
	\item \textbf{erepuscular} --
	\item \textbf{disruptive coloration} --
	\item \textbf{swim bladder} --
	\item \textbf{biozone} --
	\item \textbf{neritic provinence} --
	\item \textbf{oceanic provinence} --
	\item \textbf{epipelagic zone} --
	\item \textbf{mesopelagic zone} --
	\item \textbf{bathypelagic zone} --
	\item \textbf{abyssopelagic zone} --
	\item \textbf{euphotic zone} --
	\item \textbf{disphotic zone} --
	\item \textbf{aphotic zone} --
	\item \textbf{oxygen minimum layer (OML)} -- a zone of low dissolved
		oxygen concentration that occurs at a depth of about 700 to
		1000 meters (2300 to 3280 feet)
	\item \textbf{bioluminescence} --
	\item \textbf{detritus} --
	\item \textbf{supralittoral zone} --
	\item \textbf{subneritic provinence} --
	\item \textbf{suboceanic provinence} --
	\item \textbf{littoral zone} -- the benthic zone between the highest
		and lowest spring tide shorelines, also known as the
		intertidal zone
	\item \textbf{sublittoral zone} --
	\item \textbf{bathyal zone} --
	\item \textbf{abyssal zone} --
	\item \textbf{hadal zone} --
\end{itemize}
