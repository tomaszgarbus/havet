\section{Review}

\subsection{Chapter 1}

\subsubsection{What is the technical difference between an ocean and a sea?}

Sea is defined as smaller and shallower than an ocean, composed of salt water,
somewhat enclosed by land, directly connected to world ocean.

\subsection{Chapter 2}

\subsubsection{When did Pangaea exist? What was the ocean that surrounded the
supercontinent called?}

It existed about 200 Mya. The ocean around it was called Panthalassa.

\subsection{Chapter 3}

\subsubsection{Explain how submarine canyonds are created}

Submarine canyons are carved by turbidity currents. Turbidity currents are
underwater avalanches of muddy water mixed with rocks and other debris. They
are strong enough to transport huge rocks down submarine canyons and create big
erosion over time.

\subsection{Chapter 4}

\subsubsection{Describe the origin, composition, and distribution of
hydrogenous sediment.}

Manganese nodules are hard rounded lumps of manganese, iron and other metals
formed by precipitation around some nucleus like a shark tooth.

They lie on sediment in vast expanses of abyssal plains, potentially covering
some 60\% of the ocean basin at typical water depth of 5km.

Phosphates occur as coating on rocks and as nodules on the continental shelf
and on banks shallower than 1000m.

Evaporites form wherever there are high evaporation rates (dry climates)
accompanied by restricted open ocean circulation.

\subsection{Chapter 5}

\subsubsection{Why are the freezing and boiling points of water higher than
would be expected for a compound of its molecular makeup?}

Due to unusual geometry and polarity of water molecule. Additional heat energy
is required to overcome its hydrogen bonds and van der Waals forces.

\subsection{Chapter 6}

\subsubsection{Is Earth's atmosphere heated from above or below? Explain}

From below. The Sun's energy passes trough the Earth's atmosphere and warms
the Earth's surface, which in turn radiates this energy back into the
atmosphere as heat.

\subsection{Chapter 7}

\subsubsection{How many subtropical gyres exist worldwide? How many main
currents exist within each subtropical gyre?}

There are 5 subtropical gyres, one for each ocean. Each consists of 4 main
currents.

\subsection{Chapter 8}

\subsubsection{Why is the development of internal waves likely within the
pycnocline?}

Pycnocline is a layer of rapidly changing density.

The development of internal waves is most likely there because internal waves
are created by the movement of water of different densities, also called
water-water interface.

\subsection{Chapter 9}

\subsubsection{If Earth did not have the Moon orbiting it, would there still
be tides? Why or why not?}

Yes, because the Sun's tide-generating force is about 46\% of that of the
Moon's.

\subsection{Chapter 10}

\subsubsection{Explain the difference between the shore and the coast.}

The shore is a zone that lies between the lowest tide level and the highest
elevation on land that is affected by storm waves.

The coast extends inland from the shore as far as ocean-related features can be
found.

\subsection{Chapter 11}

\subsubsection{How would dumping sewage in deeper water off the East Coast
help reduce negative impacts on the ocean floor?}

Because the dumped particles would be transported and spread by internal waves
and therefore would end up in lower density at the ocean floor, so the damage
would be less intense.

\subsection{Chapter 12}

\subsubsection{Describe the lifestyles of plankton, nekton and benthos. Why
does plankton account for a much larger percentage of the ocean's biomass
than benthos and nekton combined?}

Planton drifts in the water, nekton actively swims, benthos sits on the bottom.

Free-living bacterioplankton living off of photosynthesis constitutes half of
total marine photosynthetic biomass.

\subsection{Chapter 13}

\subsubsection{Explain why everything in the deep ocean below the shallowest
surface water appears blue-green in color}

Ocean color is influenced by the amount of turbidity from runoff (turbidity
is suspended matter in water) and the amount of photosyntetic pigment, which
increases with increasing primary productivity.

But more relevant to this question, it's because of selective filtering of
different wavelengths of light by water. Around the depth of 10m and more,
only blue, green and violet wavelengths propagate.

\subsection{Chapter 14}

\subsubsection{Are most fast-swimming fish cold-blooded or warm-blooded?
What advantage does this provide?}

Higher body temperature is associated with higher metabolic rates, which
allows them to more effectively seek and capture prey. It also speeds up
physiological process in the body, resulting in muscles that contract faster.
