\section{Review}

\subsection{Chapter 4}

\subsubsection{Describe the origin, composition, and distribution of
hydrogenous sediment.}

Manganese nodules are hard rounded lumps of manganese, iron and other metals
formed by precipitation around some nucleus like a shark tooth.

They lie on sediment in vast expanses of abyssal plains, potentially covering
some 60\% of the ocean basin at typical water depth of 5km.

Phosphates occur as coating on rocks and as nodules on the continental shelf
and on banks shallower than 1000m.

Evaporites form wherever there are high evaporation rates (dry climates)
accompanied by restricted open ocean circulation.

\subsection{Chapter 5}

\subsubsection{Why are the freezing and boiling points of water higher than
would be expected for a compound of its molecular makeup?}

Due to unusual geometry and polarity of water molecule. Additional heat energy
is required to overcome its hydrogen bonds and van der Waals forces.

\subsection{Chapter 6}

\subsubsection{Is Earth's atmosphere heated from above or below? Explain}

From below. The Sun's energy passes trough the Earth's atmosphere and warms
the Earth's surface, which in turn radiates this energy back into the
atmosphere as heat.

\subsection{Chapter 7}

\subsubsection{How many subtropical gyres exist worldwide? How many main
currents exist within each subtropical gyre?}

There are 5 subtropical gyres, one for each ocean. Each consists of 4 main
currents.

\subsection{Chapter 8}

\subsubsection{Why is the development of internal waves likely within the
pycnocline?}

Pycnocline is a layer of rapidly changing density.

The development of internal waves is most likely there because internal waves
are created by the movement of water of different densities, also called
water-water interface.

\subsection{Chapter 9}

\subsubsection{If Earth did not have the Moon orbiting it, would there still
be tides? Why or why not?}

Yes, because the Sun's tide-generating force is about 46\% of that of the
Moon's.

\subsection{Chapter 10}

\subsubsection{Explain the difference between the shore and the coast.}

The shore is a zone that lies between the lowest tide level and the highest
elevation on land that is affected by storm waves.

The coast extends inland from the shore as far as ocean-related features can be
found.
