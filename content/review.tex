\section{Review}

\subsection{Chapter 1}

\subsubsection{What is the technical difference between an ocean and a sea?}

Sea is defined as smaller and shallower than an ocean, composed of salt water,
somewhat enclosed by land, directly connected to world ocean.

\subsection{Chapter 2}

\subsubsection{When did Pangaea exist? What was the ocean that surrounded the
supercontinent called?}

It existed about 200 Mya. The ocean around it was called Panthalassa.

\subsubsection{Describe the differences between oceanic ridges and oceanic
rises. Include in your answer why these differences exist.}

Oceanic ridges are much taller and steeper. This depends on the pace of
spreading of the ocean floor. If it spreads faster (oceanic rises), then the
newly produced rocks move away from the spreading center faster and produce
less steepness.

\subsection{Chapter 3}

\subsubsection{Explain how submarine canyonds are created}

Submarine canyons are carved by turbidity currents. Turbidity currents are
underwater avalanches of muddy water mixed with rocks and other debris. They
are strong enough to transport huge rocks down submarine canyons and create big
erosion over time.

\subsubsection{Explain what graded bedding is and how it forms}

Graded bedding occurs on continental rises. As the turbidity current moves
through and erodes a submarine canyon, larger pieces settle first, then smaller
pieces later.

\subsubsection{Describe the major features of a passive continental margin:
continental shelf, continental slope, continental rise, submarine canyon,
and deep-sea fans.}

Continental shelf is a generally flat zone extending from the shore beneath
the ocean surface to the shelf break.

Continental slope  lies at the base of the shelf break. That's where the deep
ocean basins begin.

Submarine canyons are canyons in the shelf carved by the turbidity currents.

Deep-sea fans are fan-shaped (or apron-shaped) wide deposits of turbidite at
the mouths of submarine canyons.

\subsection{Chapter 4}

\subsubsection{Describe the origin, composition, and distribution of
hydrogenous sediment.}

Manganese nodules are hard rounded lumps of manganese, iron and other metals
formed by precipitation around some nucleus like a shark tooth.

They lie on sediment in vast expanses of abyssal plains, potentially covering
some 60\% of the ocean basin at typical water depth of 5km.

Phosphates occur as coating on rocks and as nodules on the continental shelf
and on banks shallower than 1000m.

Evaporites form wherever there are high evaporation rates (dry climates)
accompanied by restricted open ocean circulation.

\subsection{Chapter 5}

\subsubsection{Why are the freezing and boiling points of water higher than
would be expected for a compound of its molecular makeup?}

Due to unusual geometry and polarity of water molecule. Additional heat energy
is required to overcome its hydrogen bonds and van der Waals forces.

\subsubsection{Along the Arctic Circle, how would the Sun appear during the
summer solstice? During the winter solstice?}

On summer solstice it never goes below the horizon and on the winter solstice
it doesn't go above the horizon.

\subsubsection{Describe the Coriolis effect in both the Northern and Southern
Hemisphere}

In the Northern Hemisphere, the each object will follow a path to the right
of its intended direction. In the Southern Hemisphere, it will follow a path
to the left of its intended direction.

\subsection{Chapter 6}

\subsubsection{Is Earth's atmosphere heated from above or below? Explain}

From below. The Sun's energy passes trough the Earth's atmosphere and warms
the Earth's surface, which in turn radiates this energy back into the
atmosphere as heat.

\subsubsection{Describe differences between sea ice, icebergs, and shelf ice,
including how each is formed}

Sea ice forms directly from sea water when it's freezing into ice.
Icebergs break off from glaciers. Shelf ice occurs in Antarctica, where the
edges of glaciers form thick floating sheets of ice.

\subsection{Chapter 7}

\subsubsection{How many subtropical gyres exist worldwide? How many main
currents exist within each subtropical gyre?}

There are 5 subtropical gyres, one for each ocean. Each consists of 4 main
currents.

\subsubsection{Describe the global effects of severe El Nińos}

They alter the atmospheric jet stream and produce unusual weather in most parts
of the globe.

\subsection{Chapter 8}

\subsubsection{Why is the development of internal waves likely within the
pycnocline?}

Pycnocline is a layer of rapidly changing density.

The development of internal waves is most likely there because internal waves
are created by the movement of water of different densities, also called
water-water interface.

\subsubsection{What physical feature of a wave is related to the depth of the
wave base? What is the difference between the wave base and still water level?}

Depth of the wave equals half of its wavelength.

Still water level is the water level if there were no waves.

Wave base is the depth at which orbital motion ceases.

\subsubsection{Why is it more likely that a tsunami will be generated by the
vertical movement of sea floor faults rather than the horizontal movement
of sea floor vaults?}

Because vertical movement of sea floor changes the volume of the ocean basin
which affects the entire water column.

\subsection{Chapter 9}

\subsubsection{If Earth did not have the Moon orbiting it, would there still
be tides? Why or why not?}

Yes, because the Sun's tide-generating force is about 46\% of that of the
Moon's.

\subsubsection{Why are there tidal bulges on both sides of Earth (for example,
not just the side of Earth that faces the Moon or the Sun)?}

Because on both sides of the Earth there is a difference between the required
centrifugal forces and the actual gravitational attraction of the Moon.

\subsubsection{Explain why the Sun's influence on Earth's tides is only
46\% that of the Moon, even though the Sun is so much more massive than the
Moon.}

Because the Sun is much further away and therefore the difference in its
gravity pull on both sides of the Earth is smaller.

\subsubsection{Are tides considered deep-water waves anywhere in the ocean?
Why or why not?}

Yes there are some internal waves generated by tides.

\subsection{Chapter 10}

\subsubsection{Explain the difference between the shore and the coast.}

The shore is a zone that lies between the lowest tide level and the highest
elevation on land that is affected by storm waves.

The coast extends inland from the shore as far as ocean-related features can be
found.

\subsubsection{Describe the origin of these depositional features: spit,
bay barrier, tombolo, and barrier island}

A spit is a linear ridge of sediment that extends in the direction of longshore
drift from land into the deeper water near the mouth of a bay.

It may with time extend further to create a bay barrier.

The origin of barrier islands is complicated but is associated with melting
of glaciers at the end of the most recent ice age.

Tombolo is a sand ridge that connects and island or a sea stack to the
mainland. Tombolos can also connect two adjacent islands. Tombolos form in
the wave-energy shadow of an island and as a result are usually oriented
perpendicular to the average direction of incoming waves.

\subsubsection{Describe how an ice age affects sea level.}

During an ice age, some water is tied up as ice on land, leaving less water
for the oceans.

\subsubsection{List the two basic processes by which coasts advance seaward
and list their counterparts that lead to coastal retreat.}

erosion -- deposition

emerging -- submerging

\subsection{Chapter 11}

\subsubsection{How would dumping sewage in deeper water off the East Coast
help reduce negative impacts on the ocean floor?}

Because the dumped particles would be transported and spread by internal waves
and therefore would end up in lower density at the ocean floor, so the damage
would be less intense.

\subsection{Chapter 12}

\subsubsection{Describe the lifestyles of plankton, nekton and benthos. Why
does plankton account for a much larger percentage of the ocean's biomass
than benthos and nekton combined?}

Planton drifts in the water, nekton actively swims, benthos sits on the bottom.

Free-living bacterioplankton living off of photosynthesis constitutes half of
total marine photosynthetic biomass.

\subsubsection{What factors account for the fact that most marine species
inhabit the benthic environment?}

Because the benthic environments are much more diverse, with extreme
environmental variability, whereas pelagic environments are quite uniform.

\subsection{Chapter 13}

\subsubsection{Explain why everything in the deep ocean below the shallowest
surface water appears blue-green in color}

Ocean color is influenced by the amount of turbidity from runoff (turbidity
is suspended matter in water) and the amount of photosyntetic pigment, which
increases with increasing primary productivity.

But more relevant to this question, it's because of selective filtering of
different wavelengths of light by water. Around the depth of 10m and more,
only blue, green and violet wavelengths propagate.

\subsection{Chapter 14}

\subsubsection{Are most fast-swimming fish cold-blooded or warm-blooded?
What advantage does this provide?}

Higher body temperature is associated with higher metabolic rates, which
allows them to more effectively seek and capture prey. It also speeds up
physiological process in the body, resulting in muscles that contract faster.
