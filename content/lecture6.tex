\section{Lecture 6: Water and seawater}

Water has unusual chemical properties:
\begin{itemize}
    \item Melting point
    \item Boiling point
\end{itemize}

Atomic structure:

\begin{itemize}
    \item Nucleus contains:
    \begin{itemize}
        \item Neutrons (no charge)
        \item Protons (+ charge)
    \end{itemize}
    \item Shells contain: electrons
\end{itemize}

Every chemical element (\textit{grundämne}) is defined by the
number of protons (and an equal number of electrons)

\subsection{Different bonds}

\subsubsection{Covalent bond}

\textit{intramolecular bond} -- atoms share electrons to form molecules

\subsubsection{Ionic bond}
\textit{intramorecular bond}

\subsubsection{Hydrogen bond}

\subsection{The water (H$_2$O) molecule has covalent bonds between
H and O atoms}

\begin{itemize}
    \item Water is composed of 1 oxygen atom and 2 hydrogen atoms.
    \item Contains covalent bonds between oxygen and hydrogen atoms.
    \item Covalent bonds are when atoms share electrons.
\end{itemize}

\subsection{Hydrogen bonds between water molecules}

\begin{itemize}
    \item Polarity causes water molecules to form weak (hydrogen)
        bonds between water molecules since the + and - sides are
        attracted to each other.
\end{itemize}

\subsection{Water is a solvent}

Sodium and chlorine form ionic bonds. The sodium atom
gives an electron to the chlorine atom. It is a relatively weak bond.

\subsection{Water exists in the 3 states of matter (solid, liquid and gas)}

\begin{itemize}
    \item Latent (hidden) heat = energy that is either absorbed or released
        from water as the water changes state.
    \item Heat is absorbed when chemical bonds break.
    \item Heat is released when bonds are created.
\end{itemize}

\subsection{Hydrogen bonds in H$_2$O and the three states of matter}

In the solid state there are hydrogen bonds between all water molecules.

In the liquid state there are some hydrogen bonds.

In the gas state, there are no hydrogen bonds, and the water molecules move
rapidly and independently.

\subsection{Snowflake geometry}

Snowflake geometry is caused by the geometry of water crystalline structure.

\subsection{Latent heats and changes of state of water}

Latent heat of melting: 80cal/g

Latent heat of vaporization: 540cal/g

\subsection{Comparison of melting and boiling points of water with similar
chemical compounds}

Water has much higher melting and boiling points thanks to having a
130$\degree$ angle between hydrogen molecules (compared to a similar
molecule if it had a 180$\degree$ angle between molecules).

\subsection{Heat capacity}

\begin{itemize}
    \item Heat capacity is the amount of energy required to raise the
        temperature of 1 gram of material 1C.
    \item Water has high heat capacity compared to other natural
        materials.
\end{itemize}

\subsection{Ocean water stabilizes Earth's temperature change}

\begin{itemize}
    \item Water has high heat capacity, so it can absorb (or release)
        large quantities of energy with slow changes of temperature
    \item Ocean water moderates coastal temperatures
\end{itemize}

\subsection{Surface tension}

\begin{itemize}
    \item Due to the polarity, water molecules want to cling
        to each other.
    \item At the surface, the outmost layer of molecules, has fewer
        molecules to cling to.
    \item Molecules compensate by establishing stronger bonds with its
        neighbours -- this leads to the formation of the surface tension
    \item Other than mercury, water has the greatest surface tension of
        any liquid
\end{itemize}

\subsection{Special properties of water due to polarity}

\begin{itemize}
    \item High surface tension
    \item Good solvent
    \item High heat capacity
    \item Latent heat of melting (80 cal)
    \item Latent heat of vaporization (540 cal)
    \item Density -- water density increases with decreasing temperature
        BUT then decreases below 4$\degree$ C
    \item Thermal expension -- Contracts as it cools BUT expands
        below 4$\degree$
    \item Exists in 3 states on the surface of the earth: ice, liquid, gas
\end{itemize}

\subsection{Sea water -- composition of ocean salt}

\subsection{Salinity}

\begin{itemize}
    \item Salinity = total amount of solid material dissolved in water
    \item Can be determined by measuring water conductivity (how easy
        it is for electricity to flow)
    \item Typically expressed in parts per thousand
\end{itemize}

\subsection{Surface salinity variation}

\begin{itemize}
    \item Pattern of surface salinity:
        \begin{itemize}
            \item Lowest in high latitudes
            \item Highest in the tropics
            \item Decrease at the Equator
        \end{itemize}
    \item Surface processes help explain pattern
\end{itemize}

\subsection{Salinity variation with depth}

\begin{itemize}
    \item Measurements from different latitudes have different surface
        salinities
    \item Halocline = layer of rapidly changing salinity
    \item At depth, salinity is relatively uniform
\end{itemize}

\subsection{Seawater density}

\begin{itemize}
    \item Factors affecting seawater density:
        \begin{itemize}
            \item Temperature increases, density decreases
            \item Salinity increases, density increases
            \item Pressure increases, density increases
        \end{itemize}
    \item Temperature has the greatest influence on surface seawater
        density
\end{itemize}

\subsection{Temperature and density variations with depth}

\begin{itemize}
    \item \textbf{Pycnocline} = layer of strong changing density
    \item \textbf{Thermocline} = layer of strong changing temperature
    \item The pycnocline is a barrier to vertical mixing of water
        and migration of marine life.
    \item Isopycnal water columns allow water to vertically mix.
\end{itemize}

\subsection{Ocean layering based on density}

\begin{itemize}
    \item Mixed surface layer (surface to 300 meters):
        low density; well mixed by waves, currents, tides
    \item Upper water (300 to 1000m): intermediate density
        water containing thermocline, pycnocline, and halocline
    \item Deep water (below 1000m): cold, high density water
        involved in deep current movement
\end{itemize}

\subsection{Distillation}