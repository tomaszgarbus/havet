\section{Chapter 1}

\subsection{Oceans}

Pacific Ocean:
\begin{itemize}
	\item largest ocean
	\item more than hald of ocean surface on Earth
	\item over one third of Earth's entire surface
	\item deepest ocean
\end{itemize}

Atlantic Ocean:
\begin{itemize}
	\item about half the size of Pacific Ocean
	\item separates the Old World (Europe, Asia, Africa) from the New World
		(North and South America)
	\item named after Atlas, one of the Titans in Greek mythology
\end{itemize}

Indian Ocean:
\begin{itemize}
	\item slighty smaller than Atlantic Ocean
	\item about same average depth as Atlantic
	\item mostly in the Southern Hemisphere
\end{itemize}

Arctic Ocean:
\begin{itemize}
	\item about 7\% size of the Pacific Ocean
	\item only a bit over one-quarter as deep as the rest of the oceans
	\item has a permanent layer of sea ice at the surface, but the ice is
		only a few meters thick
\end{itemize}

Southern or Antarctic Ocean:
\begin{itemize}
	\item it is really the portions of Pacific, Atlantic and Indian oceans
		south of about 50 degrees south latitude
\end{itemize}

The average depth of the world's oceans is 3682 meters. The deepest depth in
the oceans (the Challenger Deep region of the Mariana Trench) is 11022 meters
below sea level.

The average height of the continents is only 840 meters.

\subsection{History of Ocean exploration}

\begin{itemize}
	\item Pytheas in 325 B.C. sailed northward using a simple method for
		determining latitude in the Northern Hemisphere
	\item Eratosthenes used the shadow of a stick in a hole in the ground
		and elementary geometry to determine Earth's circumference to
		be 40000 km
	\item Claudius Ptolemy produced a map of the world in about 150 A.D.
		that represented the extent of Roman knowledge at that time
	\item late in the 10th century the Vikings colonized Iceland
	\item in about 981 A.D. Erik "the Red" Throvaldson sailed westward
		from Iceland and discovered Greenland
	\item Leif Eriksson, son of Erik the Red, found Vinland (Newfoundland,
		Canada) and spent the winter there
	\item 1492 to 1522 is known in Europe as \textit{Age of Discovery}.
		Southern Europeans explored the continents of South and
		North America then.
	\item Captain James Cook blah blah
\end{itemize}

\subsection{What is Oceanography}

It is an \textbf{interdisciplinary science}

Geology:
\begin{itemize}
	\item sea floor tectonics
	\item coastal processes
	\item sediments
	\item hydrologic cycle
\end{itemize}

Geography:
\begin{itemize}
	\item wind belts
	\item weather
	\item coastal landforms
	\item world climate
\end{itemize}

Biology:
\begin{itemize}
	\item fisheries
	\item ecological surveys
	\item microbiology
	\item marine adaptations
\end{itemize}

Chemistry:
\begin{itemize}
	\item dissolved components
	\item temperature dependence
	\item stratification/density
	\item chemical tracers
\end{itemize}

Physics:
\begin{itemize}
	\item currents
	\item waves
	\item sonar
	\item thermal properties of water
\end{itemize}

Astronomy:
\begin{itemize}
	\item tidal forces
	\item oceans on other planets
	\item origin of water
	\item origin of life
\end{itemize}

\subsection{Glossary of terms}

\begin{itemize}
	\item \textbf{oceans} -- 
\end{itemize}
