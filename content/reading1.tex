\section{Chapter 1}

\subsection{Oceans}

Pacific Ocean:
\begin{itemize}
	\item largest ocean
	\item more than hald of ocean surface on Earth
	\item over one third of Earth's entire surface
	\item deepest ocean
\end{itemize}

Atlantic Ocean:
\begin{itemize}
	\item about half the size of Pacific Ocean
	\item separates the Old World (Europe, Asia, Africa) from the New World
		(North and South America)
	\item named after Atlas, one of the Titans in Greek mythology
\end{itemize}

Indian Ocean:
\begin{itemize}
	\item slighty smaller than Atlantic Ocean
	\item about same average depth as Atlantic
	\item mostly in the Southern Hemisphere
\end{itemize}

Arctic Ocean:
\begin{itemize}
	\item about 7\% size of the Pacific Ocean
	\item only a bit over one-quarter as deep as the rest of the oceans
	\item has a permanent layer of sea ice at the surface, but the ice is
		only a few meters thick
\end{itemize}

Southern or Antarctic Ocean:
\begin{itemize}
	\item it is really the portions of Pacific, Atlantic and Indian oceans
		south of about 50 degrees south latitude
\end{itemize}

The average depth of the world's oceans is 3682 meters. The deepest depth in
the oceans (the Challenger Deep region of the Mariana Trench) is 11022 meters
below sea level.

The average height of the continents is only 840 meters.

\subsection{Glossary of terms}

\begin{itemize}
	\item \textbf{oceans} -- 
\end{itemize}
