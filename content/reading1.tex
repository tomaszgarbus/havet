\section{Chapter 1}

\subsection{Oceans}

Pacific Ocean:
\begin{itemize}
	\item largest ocean
	\item more than hald of ocean surface on Earth
	\item over one third of Earth's entire surface
	\item deepest ocean
\end{itemize}

Atlantic Ocean:
\begin{itemize}
	\item about half the size of Pacific Ocean
	\item separates the Old World (Europe, Asia, Africa) from the New World
		(North and South America)
	\item named after Atlas, one of the Titans in Greek mythology
\end{itemize}

Indian Ocean:
\begin{itemize}
	\item slighty smaller than Atlantic Ocean
	\item about same average depth as Atlantic
	\item mostly in the Southern Hemisphere
\end{itemize}

Arctic Ocean:
\begin{itemize}
	\item about 7\% size of the Pacific Ocean
	\item only a bit over one-quarter as deep as the rest of the oceans
	\item has a permanent layer of sea ice at the surface, but the ice is
		only a few meters thick
\end{itemize}

Southern or Antarctic Ocean:
\begin{itemize}
	\item it is really the portions of Pacific, Atlantic and Indian oceans
		south of about 50 degrees south latitude
\end{itemize}

The average depth of the world's oceans is 3682 meters. The deepest depth in
the oceans (the Challenger Deep region of the Mariana Trench) is 11022 meters
below sea level.

The average height of the continents is only 840 meters.

\subsection{History of Ocean exploration}

\begin{itemize}
	\item Pytheas in 325 B.C. sailed northward using a simple method for
		determining latitude in the Northern Hemisphere
	\item Eratosthenes used the shadow of a stick in a hole in the ground
		and elementary geometry to determine Earth's circumference to
		be 40000 km
	\item Claudius Ptolemy produced a map of the world in about 150 A.D.
		that represented the extent of Roman knowledge at that time
	\item late in the 10th century the Vikings colonized Iceland
	\item in about 981 A.D. Erik "the Red" Throvaldson sailed westward
		from Iceland and discovered Greenland
	\item Leif Eriksson, son of Erik the Red, found Vinland (Newfoundland,
		Canada) and spent the winter there
	\item 1492 to 1522 is known in Europe as \textit{Age of Discovery}.
		Southern Europeans explored the continents of South and
		North America then.
	\item Captain James Cook blah blah
\end{itemize}

\subsection{What is Oceanography}

It is an \textbf{interdisciplinary science}

Geology:
\begin{itemize}
	\item sea floor tectonics
	\item coastal processes
	\item sediments
	\item hydrologic cycle
\end{itemize}

Geography:
\begin{itemize}
	\item wind belts
	\item weather
	\item coastal landforms
	\item world climate
\end{itemize}

Biology:
\begin{itemize}
	\item fisheries
	\item ecological surveys
	\item microbiology
	\item marine adaptations
\end{itemize}

Chemistry:
\begin{itemize}
	\item dissolved components
	\item temperature dependence
	\item stratification/density
	\item chemical tracers
\end{itemize}

Physics:
\begin{itemize}
	\item currents
	\item waves
	\item sonar
	\item thermal properties of water
\end{itemize}

Astronomy:
\begin{itemize}
	\item tidal forces
	\item oceans on other planets
	\item origin of water
	\item origin of life
\end{itemize}

Four main disciplines of oceanography:

\begin{itemize}
	\item \textbf{geological} oceanography
	\item \textbf{chemical} oceanography
	\item \textbf{physical} oceanography
	\item \textbf{biological} oceanography
\end{itemize}

\subsection{Density stratification}

\begin{itemize}
	\item once Earth became a ball of hot liquid rock, the elements were
		able to segregate according to their densities in a process
		called density stratification
	\item highest-density materials (primarily iron and nickel)
		concentrated in the core
	\item progressively lower-density components (primarily rocky material)
		formed concentric spheres around the core
\end{itemize}

Earth consists of 3 chemical layers:
\begin{itemize}
	\item crust: about 30km deep
	\item mantle: about 2885km deep
	\item core: to the center of the Earth at 6371km deep
\end{itemize}

Earth consists of 5 physical layers:
\begin{itemize}
	\item inner core: rigid and does not flow (because of increased
		pressure at the center of Earth)
	\item outer core: liquid and capable of flowing
	\item mesosphere: extends from 700km to 2885km deep, which
		corresponds to the middle and lower mantle.
		It is rigid due to increased pressure at these depths.
	\item asthenosphere: plastic (flows under force). Extends from about
		100km to 700km deep. Hot enough to partially melt portions
		of most rocks. Corresponds to the base of the upper mantle.
	\item litosphere: cool rigid outermost layer. Avg depth 100km.
		Includes the crust plus the topmost portion of the mantle.
\end{itemize}

\subsection{Oceanic vs continental crust}

\textbf{Oceanic crust}
\begin{itemize}
	\item Oceanic crust underlies the ocean basins and is composed of
		basalt and has 3x higher density than water.
	\item The avg thickness of the oceanic crust is about 8km.
	\item Basalt originates as molten magma beneath Earth's crust
		(typically from the mantle), some of which comes to surface
		during underwater sea floor eruptions.
\end{itemize}

\textbf{Continental crust}
\begin{itemize}
	\item Composed mostly of lower-density and lighter-colored igneous
		rock granite
	\item It has density of about 2.7g per cubic cm
	\item The avg thickness of the continental crust is about 35km, up to
		60km beneath the highest mountain ranges.
	\item Most granite originates beneath the surface as molten magma that
		cools and hardens within Earth's crust.
\end{itemize}

No matter which type of crust is at the surface, it is all part of the
litosphere.

\subsection{Asthenosphere}

\begin{itemize}
	\item Relatively hot, plastic region beneath the litosphere.
	\item Extends from the base of the litosphere to about 700km.
	\item Entirely contained within the upper mantle.
	\item Can deform without fracturing if a force is applied slowly.
	\item High viscosity (stickiness, resistance to flow).
\end{itemize}

\subsection{Isostatic adjustment}

\begin{itemize}
	\item The vertical movement of crust is the result of the buoyancy of
		Earth's lithospere as it floats on the denser, plastic-like
		asthenosphere below.
	\item For example, a heavier ship  with more cargo will sit lower in
		the water than a lighter ship.
	\item Similarly, both continental and oceanic crust float on the denser
		mantle beneath and get adjusted.
	\item \textbf{Isostatic rebound} -- example is how the sea floor is 
		rising in Scandinavia after the last ice age
\end{itemize}

\subsection{How were Earth's atmosphere and oceans formed?}
\begin{itemize}
	\item Early in Earth's history, volcanic activity released large
		amounts of water vapour into the atmosphere.
	\item Water vapour condensed into clouds.
	\item Liquid water fell to Earth's surface where it accumulated in
		low areas and over time formed the oceans.
\end{itemize}

Atmosphere:

\begin{itemize}
	\item Earth's initial atmosphere consisted of leftover gases from the
		nebula.
	\item \textbf{Outgassing} -- expelling of gasses from inside Earth
\end{itemize}

Oceans:
\begin{itemize}
	\item Since outgassing releases mostly water vapour, this was the
		primary source of water on Earth, including supplying oceans
		with water.
	\item Water could have also been supplied by space debris leftovers
		(from the forming of the solar system) bombarding the Earth.
	\item Development of ocean salinity: early atmosphere had a lot of
		carbon and sulfur dioxide content which created very acidic
		rain, capable of dissolving greater amounts of minerals in the
		crust than occurs today.
\end{itemize}

\subsection{Did life begin in the oceans?}

\begin{itemize}
	\item Earliest-known life-forms were primitive bacteria that lived in
		sea floor rocks about 3.5 Bya.
	\item Oxygen is essential to human life for 2 reasons:
		\begin{itemize}
			\item Our bodies need oxygen to "burn" (oxidize)
				food, releasing energy to our cells.
			\item Oxygen in the upper atmosphere in the form of
				ozone protects the surface of Earth from most
				of the Sun's harmful ultraviolet radiation.
		\end{itemize}
\end{itemize}

\subsection{Glossary of terms}

\begin{itemize}
	\item \textbf{oceans} -- the entire body of saltwater that covers
		70\% of Earth's surface
	\item \textbf{nebula} -- a huge cloud of gas and space dust
	\item \textbf{nebular hypothesis} -- all bodies in the solar system
		formed from an enormous cloud composed mostly of hydrogen and
		helium with only a small percentage of heavy elements
	\item \textbf{Pacific Ocean} -- The ocean located between Australia,
		Asia, North America, and South America; the largest ocean in
		the world.
	\item \textbf{Atlantic Ocean} -- The ocean located between South
		America, North America, Europe, and Africa; the second largest
		ocean in the world.
	\item \textbf{Indian Ocean} -- The ocean located between Africa, India,
		and Australia; it exists mostly in the Southern Hemisphere
		and is the third largest ocean in the world.
	\item \textbf{Arctic Ocean} -- The ocean located in the Northern
		Hemisphere polar region; the smallest ocean in the world.
	\item \textbf{Southern Ocean / Antarctic Ocean} -- The ocean that
		surrounds the continent of Antarctica and is located south of
		about 50 degrees south latitude.
	\item \textbf{Latitude} -- Location on Earth's surface based on angular
		distance north or south of the equator. Equator = 0 degrees.
		North Pole = 90 degrees north. South Pole = 90 degrees south.
	\item \textbf{Longitude} -- Location on Earth's surface based on
		angular distance east or west of the Prime (Greenwhich)
		Meridian (0 degrees longitude). 180 degrees longitude is the
		International Date Line.
	\item \textbf{nebular hypothesis} -- A model that describes the
		formation of the solar system by contraction of a nebula.
	\item \textbf{protoplanet} -- Any planet that is in its early stages of
		development.
	\item \textbf{thermonuclear fusion} -- A high temperature process in
		which hydrogen atoms are converted to helium atoms, thereby
		releasing large amounts of energy.
	\item \textbf{density stratification} -- A layering based on density,
		where the highest density material occupies the lowest space.
	\item \textbf{crust} -- (1) The uppermost outer layer of Earth's
		structure that is composed of basaltic oceanic crust and
		granitic continental crust. The average thickness of the
		crust ranges from 8km beneath the ocean to 35km beneath the
		continents. (2) A hard covering or surface layer of
		hydrogenous sediment.
	\item \textbf{mantle} -- (1) The zone between the core and crust of
		Earth; rich in ferromagnesian minerals. (2) In pelecypods,
		the portion of the body that secretes shell material.
	\item \textbf{core} -- (1) The deep, central layer of Earth, composed
		primarily of iron and nickel. It is subdivided into a liquid
		outer core 2270km thick and solid inner core with a radius of
		1216km. (2) A cylinder of sediment and/or rock material
		usually obtained by drilling.
	\item \textbf{mesosphere} -- The middle region of Earth below the
		asthenosphere and above the core.
	\item \textbf{asthenosphere} -- A plastic layer in the upper mantle
		80 to 200km deep that may allow lateral movement of
		litospheric plates and isostatic adjustments.
	\item \textbf{lithosphere} -- The outer layer of Earth's structure,
		including the crust and the upper mantle to a depth of about
		200km. Lithospheric plates are the major components involved
	\item \textbf{oceanic crust} -- A mass of rock with a basaltic
		composition that is about 5km thick.
	\item \textbf{basalt} -- A dark-colored volcanic rock characteristic of
		the ocean crust. Contains minerals with relatively high iron
		and magnesium content.
	\item \textbf{granite} -- A light-colored igneous rock characteristic
		of the continental crust that is rich in nonferromagnesian
		minerals such as feldspar and quartz.
	\item \textbf{viscosity} -- A property of a substance to offer
		resistance to flow caused by internal friction.
	\item \textbf{isostatic adjustment} -- The adjustment of crustal
		material due to isostasy.
	\item \textbf{isostatic rebound} -- The upward movement of crustal
		material due to isostasy.
	\item \textbf{outgassing} -- The process by which gases are removed
		from within the Earth's interior.
	\item \textbf{heterotrophs} -- Animals and bacteria that depend on the
		organic compounds produced by other organisms as food.
		Organisms not capable of producing their own food by
		photosynthesis.
	\item \textbf{autotrophs} -- Algae, plants, and bacteria that can
		synthesize organic compounds from inorganic nutrients.
	\item \textbf{anaerobic} -- Requiring or occuring in the absence of
		free oxygen (O$_2$).
	\item \textbf{chemosynthesis} -- A process by which bacteria or
		archaea synthesize organic molecules from inorganic nutrients
		using chemical energy released from the bonds of a chemical
		compound (such as hydrogen sulfide) by oxidation.
	\item \textbf{chlorophyll} -- A group of green pigments that make it
		possible for plants to carry on photosynthesis.
\end{itemize}
