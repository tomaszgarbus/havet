\section{Lecture 10: Marine life}

\subsection{Marine environments}

\textbf{Neritic zone}: shallow shelf zone, where light reaches to seafloor,
abdundant food, abundant life

Benthic zone: seafloor

\begin{itemize}
	\item epipelagic zone: 0-100m (continental shelf)
	\item mesopelagic zone: 100-1000m
	\item bathypelagic zone: 1000m-4000m
	\item abyssal zone: 4000-6000m
	\item deep-sea trenches and hadal zone
\end{itemize}

\subsection{Classification of marine life}

protozoa = unicellural organisms

prokaryota (bacteria and archaea) = unicellural organisms, without a membrane
bound nucleus

eukaryota = unicellural and multicellural organisms with a membrane-bound
nucleus: plantae, chromista, protista, fungi, animalia

\subsection{Classification of marine life: 3 categories}

\begin{itemize}
	\item plankton: unable to swim: float
	\item nekton: active swimmers
	\item benthos: bottom living
\end{itemize}

\subsection{Marine ecosystems = aquatic ecosystems that have high salt
content}

Nearshore (neritic) ecosystems:
\begin{itemize}
	\item saltmarshes
	\item mudflats
	\item sea-grass meadows
	\item mangroves
	\item rocky intertidal systems/kelp forests
	\item coral reefs
\end{itemize}

Offshore ecosystems:
\begin{itemize}
	\item surface ocean: pelagic ocean waters
	\item ocean floor:
	\begin{itemize}
		\item deep benthic habitats, typical seafloor
		\item oceanic hydrothermal vents and cold seeps
	\end{itemize}
\end{itemize}

Ca. 60\% of the Earth's surface is in the pelagic zone: therefore,
phytoplankton primary production is very important to global primary production
and carbon cycling

\subsection{Nekton (swimmers)}

\begin{itemize}
	\item Animals that move independently
	\item Adult fish, marine mammals, marine reptiles, some marine
		invertebrates (squid)
	\item Temperature, salinity, viscosity, pressure, nutrients influence
		where the nekton can live
	\item Fish are most common near continents, islands and in cold water.
	\item Some fish leave the ocean to spawn in fresh water (salmon).
		Some eels move from fresh water to the ocean to spawn.
	\item Jellyfish and other species are considered plankton when they
		are small/juvenile and nekton when they are larger
\end{itemize}

\subsection{Benthos (bottom dwellers)}

\begin{itemize}
	\item Organisms that live on or in the ocean bottom.
	\item Epifauna: organisms live on the surface, either attached or
		moving on the bottom.
	\item Infauna: organisms live within sand or mud of the sea bottom.
	\item Nektobenthos: some animals can swim or crawl on the bottom.
\end{itemize}

\begin{itemize}
	\item Shallow benthos survive on photosynthetic biomass growing on the
		bottom substrate. Therefore, benthic biomass decreases with
		increasing depth due to decreasing sunlight and therefore
		decreasing vegetable food.
	\item In deeper water offshore, benthic animals eat detritus falling
		from the surface ocean or continent, or each other
		(carnivorous).
	\item In the deep ocean (>2000m) where no light penetrates, most
		organisms are dependent on detritus food sources raining down
		from the surface (<0.1\% of total primary production reaches
		the seafloor). Food is scarce, therefore deep benthic biomass
		is very low biomass.
\end{itemize}

\subsection{Marine biodiversity: oceans vs land}
\begin{itemize}
	\item Fewer species are found in the oceans compared to land (lower
		diversity in the oceans).
	\item This is because the oceans are more homogenous environments,
		they are well connected by ocean currents, and well "mixed"
		by wind, waves and currents, meaning there are fewer places
		or habitats for organisms to isolate and speciate.
	\item In contrast, on land there is much greater variety in the types
		of environments providing much more ecospace and different
		habitats for organisms to evolve and adapt into.
	\item Marine species make only 14\% of the 1800000 world species.
\end{itemize}

\subsection{Benthic species are far more diverse than pelagic species}

This is because the pelagic ocean beneath the penetration of sunlight is most
homogenous, well mixed, therefore limited ecological niches.

In contrast, the ocean floor (benthic environment) is more diverse, especially
closer to land, e.g. rocky, sandy, muddy, slopes, etc., which requires
adaptations to live and creates diverse ecospace.

On the other hand, pelagic species (making up only 2\% of 250000 marine
species 98\% of benthic), have a much bigger total biomass.

\subsection{Adaptations to life in salty water: a dense fluid}
\begin{itemize}
	\item Sea water is 827x more dense than air.
	\item This combined with other physical, chemical and biological
		factors, presents challenges, and opportunities to life in
		the seas.
	\item Some of the physical conditions and challenges of living in the
		marine realm are:
	\begin{itemize}
		\item Viscous fluid
		\item Salt (chemistry challenges)
		\item Drinking (not drinking too much salt)
		\item Stay in the photic zone (where food is) and not sink
		\item Breathing in and out
		\item Dealing with high pressure
		\item Fighting against the winds, tides and currents to stay
			where you want to be
		\item Reproduction
		\item Communicating
		\item Seeing
		\item Hearing
		\item Staying warm or cool
	\end{itemize}
\end{itemize}

\subsection{Life in a viscious fluid}

Body support: sea water provides buoyance, therefore bodies are better
supported in water. This allows ocean organisms to be huge, like the blue
whale.

Size also matters for small organisms:
\begin{itemize}
	\item Plankton are usually small, small enough that internal viscosity,
		friction forces of water dominate the behaviour of the water
		acting on them. Therefore they can stay in the photic zone
		without doing much: rely on friction and natural buoyance to
		stay where they need to be.
	\item Nekton are larger and interact with their environment at high
		Reynolds numbers where turbulent flow and inertial forces
		dominate. They must actively swim to do what they need to do.
\end{itemize}

\subsection{Life in a viscous fluid: low Reynold Numbers}

Low Reynolds Number means that mostly internal forces dominate and organisms
can stay in place by affecting how the water flows around them.

\subsection{Life in a viscous fluid: high Reynold Numbers}

Larger organisms have high Reynolds numbers, therefore optimize their swimming
speed by having a specially modified body shape that helps displace water
ahead and move it behind itself.

Streamlining = the animals evolve a fish shape with a flattened body,
small cross section at the front, and tapered tail. Fish shaped body!

\subsection{Surviving in salt water}

Adaptation to salinity: saltwater fish

\begin{itemize}
	\item Saltwater fish adapt by constantly drinking water and expelling
		salt using chloride-releasing cells in the gills.
	\item The saltwater fish also discharge very small amounts of
		concentrated urine.
	\item It would be very damaging for a salt water fish to go into
		freshwater: it would drink too much fresh water and risk
		rupturing its cells.
	\item Whales, dolphins and other sea-dwelling mammals: do not drink
		seawater. They obtain water from their food and by producing
		it internally from the metabolic breakdown of food.
\end{itemize}

\subsection{Oxygen intake and breathing in the marine environment}
\begin{itemize}
	\item Single-celled organisms and jellyfish absorb O$_2$ and exchange
		CO$_2$ across their cell membranes
	\item Marine mammals and bird and some air-breathing fish, hold their
		breath then periodically come up for air.
	\item Most fish and invertebrates get their oxygen in dissolved form
		from water.
\end{itemize}

\subsection{Ocean biological productivity, food chains and energy transfer}

\begin{itemize}
	\item The oceans are responsible for over 40\% of global primary
		production.
	\item Therefore, they play a major role in the global carbon cycle
		by consuming CO$_2$ from the atmosphere and positively
		influencing climate change (CO$_2$ sink).
	\item Microalgaea are responsible for over 50\% of this production,
		with macroalae (kelp forests) contributing the rest.
	\item The oceans are a major source of food.
\end{itemize}

\subsection{Biological productivity and energy transfer}
\begin{itemize}
	\item Primary production is not equally spread out. There are ocean
		deserts and hotspots.
	\item In the open ocean, the highest zones of biological production
		coincide with regions oceanic upwelling, where nutrients are
		brought to the surface.
\end{itemize}

\subsection{Ocean biological productivity: geography and oceanography}

\begin{itemize}
	\item The low latitudes have a permanent thermocline that prevents
		upwelling, therefore nutrients are scarce in the surface
		ocean = low surface production.
	\item Equatorial upwelling, Southern Ocean, North Pacific, coastal
		and other upwelling regions, are exceptions = strong primary
		production.
	\item Shallow, tropical environments: coral reefs have low nutrients.
		Rely on symbiotic algae living with the coral to produce
		biomass and build strong ecosystems = strong primary
		production.
	\item Climatic warming causes stronger stratification, reduces mixing
		and results in lower ocean O$_2$ content which is damaging to
		marine ecosystems: O$_2$ has dropped by 2$\%$ in the last
		50 years.
\end{itemize}

\subsection{The future looks good for jelly fish}
\begin{itemize}
	\item Very long geological history: 600 million years = resilient
	\item Their increasing abundance has sparked increasing research
	\item Key question: what other animals eat jellyfish considering they
		are 95\% water?
	\item Answer: jellyfish are regular food: low calorie, but abundant
		and easy to digest.
	\item Jelly fish are environmentally tolerant, also overfishing has
		led to decreasing populations of jellyfish predators and
		competitors for food, thus helping populations rise.
	\item And there could be a pay off: although mostly water, jellyfish
		accumulate carbon: large populations, large carbon sink to help
		sequester carbon in the deep sea (and diminish CO$_2$ increase)
\end{itemize}

Meanwhile:
\begin{itemize}
	\item Increased jellyfish abundance close to shore, is pest: painful
		strings to swimmers.
	\item Problems for industry: can stop nuclear power plants from
		functioning because jellies clog cooling filters, which can
		require plants to shut down.
	\item This has happened all over the world, including Scotland's
		Torness power plant, Oskarshamn power plant in Sweden,
		Japan, Israel.
\end{itemize}

\subsection{Summary and closing remarks}
\begin{itemize}
	\item Marine organisms are diverse occupying planktonic and benthic
		habitats.
	\item Salt water presents many challenges to life in the oceans,
		requiring many adaptations.
	\item Ocean food chains start with microscopic algae and are
		responsible for 40\% global primary production: important
		for global carbon cycling.
	\item Plankton and other marine organisms are suffering due to
		climate change.
	\item Warming oceans are causing increased stratification with less
		mixing meaning that oxygen becomes depleted, suffocating
		plankton and other organisms.
	\item This is concerning because plankton are at the base of food webs,
		supporting fisheries and all the other diverse organisms.
\end{itemize}
