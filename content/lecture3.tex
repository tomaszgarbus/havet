%####&&&&&%####&&&&&%####&&&&&%####&&&&&%####&&&&&%####&&&&&%####&&&&&%####&&&&&
\section{Lecture 3: Plate Tectonics and the Ocean Floor}

\subsection{Plate tectonics: the history}

\begin{itemize}
    \item Alfred Wegener (1880-1930)
    \item German meteorologist and geophysist
    \item First with the idea of moving continents
    \item Published \textit{The origin of the continents and oceans} 1915
    \item He saw the continents as drifting: \textit{Continental drift}
\end{itemize}


\subsection{Supporting evidence for plate tectonics}

\begin{enumerate}
    \item \textbf{Matching coast lines}. Wegener's fit (according to
        coastlines) was imperfect, but if you
        use 2000m isobaths it's much better.
\end{enumerate}

\subsection{Critique to the continental drift theory}

\begin{itemize}
    \item Wegener had suggested that the continents plowed through the ocean
        basins, but how is that possible?
    \item The force was Earth's gravitational attraction from the equator
        bulge and tidal forces from the Sun and Moon. But the forces were shown
        not to be enough.
\end{itemize}

\subsection{How does a rock become magnetic?}

\begin{itemize}
    \item Magnetism in a rock only occurs when a lava has cooled off enough for
        the atoms to be able to arrange for the magnetic field. The temperature
        at which it occurs is called the \textit{Curie temperature}.
    \item When solidifying a lava, the direction of the Earth's magnetic field
        is locked in the rock, that is the magnetic domains are arranged
        according to the Earth's current polarity.
\end{itemize}

Vine-Matthews-Morley Hypothesis

\subsection{Seafloor age}

The oldest oceanic crust is 280 Ma. Crust disappears in the subduction zones.

\subsection{The bathymetry of the World ocean floor}

Bruce Heezen (1924-1977, geologist) and Marie Tharp (1920-2006, geographer).

First physiographic map of the Atlantic published by Heezen and Tharp 1957

\subsection{Harry Hess put it all together}

\begin{itemize}
    \item Hess realized that seafloor's thin sediment cover (relative to the
        sediment cover on the continents) meant that the ocean floors were
        much younger than the continents.
    \item As the seabed sediment cover thinned out toward the mid-oceanic
        ridges, these were younger than the deeper deposits of the oceans.
    \item He published an article \textit{History of ocean basins} (1962)
        in which he included the idea of seafloor spreading.
\end{itemize}

\subsection{Summary}

\begin{itemize}
    \item 1957: Heezen and Tharp's physiographic map of the Atlantic show that
        spreading ridges exist
    \item 1962: Hess provided a theory for seafloor spreading, including how
        it could work with mantle convection
    \item Vine-Matthew-Moreley hypothesis explains how the magnetic field
        changes over time are encoded in seafloor
\end{itemize}

\subsection{Passive vs active continental margins}

Passive margin: no active plate boundary

Active margin: steep slope, trench, convergent, subduction

\subsection{Submarine canyons and deep-sea fans}

\begin{itemize}
    \item \textbf{Turbidity currents} (mixtures of sediment and water)
        carve \textbf{submarine canyons} into the slope and shelf.
    \item Debris from turbidity currents creates \textbf{graded bedding}
        deposits and \textbf{deep-sea fans}.
\end{itemize}

\subsection{Abyssal plains}

\begin{itemize}
    \item Deep, flat areas formed by suspension settling of fine grained
        sediment.
    \item Volcanic peaks poke through the sediment
        \begin{itemize}
            \item Abyssal hills
            \item Seamounts
            \item Tablemounts (guyots)
        \end{itemize}
\end{itemize}

\subsection{Ocean trenches}

\begin{itemize}
    \item Deepest parts of the ocean
    \item Formed by plate convergence (subduction zones -- destruction of old
        oceanic crust)
    \item Most trenches are in the Pacific Ocean
    \item Associated with volcanic arcs
        \begin{itemize}
            \item Island arc
            \item Continental arc
        \end{itemize}
\end{itemize}

%####&&&&&%####&&&&&%####&&&&&%####&&&&&%####&&&&&%####&&&&&%####&&&&&%####&&&&&