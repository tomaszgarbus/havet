%&/()%&/()%&/()%&/()%&/()%&/()%&/()%&/()%&/()%&/()%&/()%&/()%&/()%&/()%&/()%&/()
\section{Lecture 5.1: Air-Sea interaction}

\begin{itemize}
    \item Variations in solar radiation on Earth
    \item Physical properties of the atmosphere
    \item Coriolis effect impact on moving objects
    \item Global atmospheric circulation patterns
    \item Weather and climate patterns in the ocean
    \item Sea ice and icebergs
    \item Energy from wind over the oceans
\end{itemize}

\subsection{Earth's seasons}

\begin{itemize}
    \item The Earth orbits around the Sun once a year
    \item The Earth's rotation axis is tilted at 23.5$\degree$
        with respect to the ecliptic
\end{itemize}

\subsection{Uneven solar heating on Earth}

\begin{itemize}
    \item Solar energy is spread over a larger area
    \item More solar energy is reflected back into space because of the low
        angle of the sunlight
    \item Solar energy passes through more atmosphere which absorbs, scatters
        and reflects sunlight
\end{itemize}

\subsection{Earth's seasons}

\begin{itemize}
    \item NH and SH are alternately tilted toward and away from the Sun
        each year
    \item In the summer hemisphere:
        \begin{itemize}
            \item the days are longer
            \item the Sun hits the Earth at a higher angle so radiation is
                concentrated over a smaller area
            \item the higher angle also means more energy is absorbed and less
                reflected
        \end{itemize}
\end{itemize}

\subsection{Heat transport}

\begin{itemize}
    \item A net heat gain is experienced in low latitudes
    \item A net heat loss is experienced in high latitudes
    \item The equator does not get continuously hotter and the poles
        continuously cooler! Why? Heat gain and loss are balanced by oceanic
        and atmospheric circulation.
\end{itemize}

\subsection{Physical properties of the atmosphere: Composition of dry air}

Nitrogen and oxygen gas comprise 99\% of the total, with several trace gases
making up the rest; the most significant trace gas is carbon dioxide, an
important greenhouse gas.

\subsection{Physical properties of the atmopshere: Temperature}

\begin{itemize}
    \item The troposphere is lowermost 12 km of the atmosphere where most
        weather occurs
    \item Temperature in the troposphere decreases with increasing height.
    \item Outgoing long wavelength radiation from the Earth's surface
        is absorbed by GHG in the troposphere.
    \item The troposphere is primarily heated from the Earth's surface,
        not from incoming sunlight.
\end{itemize}

\subsection{Physical properties of the atmopshere: Water vapour}

\begin{itemize}
    \item Air in the troposphere contains varying amounts of water. In gas
        form, water is called water vapour.
    \item The amount of water vapour in the atmosphere depends on air
        temperature.
    \item Cool air cannot hold so much water vapor, so is typically dry.
    \item Warm air can hold more water vapor, so is typically moist.
\end{itemize}

\subsection{Physical properties of the atmopshere: Pressure}

\begin{itemize}
    \item A column of cool, dense air in the troposphere causes high pressure
        at the surface, which will lead to air diverging (moving away) at the
        surface.
\end{itemize}

\subsection{Physical properties of the atmopshere: Movement}

\begin{itemize}
    \item Air is forced from high- to low-pressure regions
    \item Moving air is called wind
\end{itemize}

\subsection{Vertical cells in the atmosphere}

\begin{itemize}
    \item The 3 vertical circulation cells:
    \begin{itemize}
        \item Hadley 0 $\degree$ -- 30 $\degree$
        \item Ferrell 30 $\degree$ -- 60 $\degree$
        \item Polar 60 $\degree$ -- 90 $\degree$
    \end{itemize}
    \item Low pressure at 0 and 60 called the equatorial low and the subpolar
        low (rising air, rain)
    \item High pressure at 30 and 90 called the subtropical high and the
    	polar high (descending air, dry)
\end{itemize}

\subsection{Wind belts of the world}

\begin{itemize}
    \item The 3 wind belts
    \begin{itemize}
        \item Easterly trade winds
        \item Prevailing westerlies
        \item Polar easterlies
    \end{itemize}
\end{itemize}

\subsection{Weather and climate patterns: Sea and land breeze}

\begin{itemize}
    \item Rocks heat up and cool down more quickly than water
    \item The air above warm rock will be warm, leading to low air pressure
    \item The air above cool water will be cool, leading to higher air pressure
    \item Air will flow at the surface from high to low pressure
    \item The situation will reverse at night because rock cools down more
        quickly
\end{itemize}

\subsection{Review questions}

\subsubsection{Why do we have seasons?}
Northern and Southern Hemispheres alternately tilt toward and away from the Sun
each year. The one tilting towards the Sun is receiving the Sun rays at a
higher angle so radiation is concentrated over a smaller area and less
energy is reflected.

\subsubsection{Why is the top of the troposphere cooler than the Earth's
surface?}
Outgoing long wavelength radiation from the Earth's surface is absorbed by
GHG in the troposphere. The troposphere is primarily heated from the Earth's
surface, not from incoming sunlight.

\subsubsection{What happens when an air column is heated? (think density and
vertical movement)}
When an air column is heated, it rises and gets less dense.

\subsubsection{Describe the global wind belts.}
There are 3 wind belts: easterly trade winds, prevailing westerlies and
polar easterlies.

\subsubsection{How does the Coriolis effect moving air or water?}
It causes moving air or water to curve to the right in NH and to the
left in SH.

\subsubsection{How do cold and warm fronts differ?}

\subsubsection{How do land and sea breezes differ?}
In a sea breeze, the land warms faster during the day and warm air moves up,
and cold air from the sea flows over the land due to low pressure.

In a land breeze, the land air cools faster than sea air, so it moves down
again and pushes towards the sea. Warmer air at the sea moves up then.

\subsubsection{How do sea-ice and ice-bergs form?}
Sea ice forms in freezing water. It leaves behind salty water that sinks to
depth in polar regions and help form deep water.

Icebergs form when land-based fresh-water glacial ice breaks off into the sea.

\section{Lecture 5.1: Ocean currents}

\begin{itemize}
    \item Surface currents
        \begin{itemize}
            \item Affect upper 1000 m in the ocean
            \item Driven by major wind belts of the world
        \end{itemize}
    \item Deep currents
        \begin{itemize}
            \item Affect deep water below 1000 m depth
            \item Driven by sea water density differences
            \item Larger and slower than surface currents
        \end{itemize}
\end{itemize}

\subsection{Surface currents closely follow global wind belt pattern}

\begin{itemize}
    \item Trade winds at 0$\degree$-30$\degree$ latitude
        blow surface ocean water to the west.
    \item Prevailing westerlies at 30$\degree$-60$\degree$
        latitude blow surface ocean water from west to east
\end{itemize}

\subsection{Upwelling and downwelling}

\begin{itemize}
    \item Vertical surface layer movement (some 100s m depth)
        \begin{itemize}
            \item Upwelling = movement of underlying water to the surface
                \begin{itemize}
                    \item Lifts cold, nutrient-rich water to surface
                    \item Produces high primary productivities and abundant
                        marine life
                \end{itemize}
        \end{itemize}
\end{itemize}

\subsection{Review questions}

\subsubsection{What is the difference between surface and deep currents}
Surface currents are in upper 1000m in the ocean and deep currents are below
the first 1000m.

Surface currents are driven by major wind belts, deep currents are driven
by sea water density differences.

Deep currents are larger and slower than surface currents.

\subsubsection{Describe the make up of a subtropical gyre}
Gyres are large circular-moving loops of water.

Subtropical gyres are centered around 30$\degree$N or 30$\degree$S.

There are five main, one in each ocean basic: North Pacific, South Pacific,
North Atlantic, South Atlantic, Indian.

\subsubsection{How is the eastern and western side of a subtropical gyre
different}
Western boundary currents of subtropical gyres are:

\begin{itemize}
	\item faster (100km/day vs 10km/day for eastern)
	\item narrow (under 100km wide vs 1000km for eastern)
	\item deep (up to 2km deep vs under 0.5km for eastern)
	\item warm (vs eastern are cold)
\end{itemize}

\subsubsection{How does the wind drive ocean crrents? Describe the Ekman
spiral}
The Ekman spiral is driven by the wind. Due to water's resistance, the current
is only at 45$\degree$ from the direction of the wind (to the right in NH)
and deeper in the water the direction is deflected even more, turning into a
spiral.

\subsubsection{Can you tell fro a sea surface temperature plot whether the
Pacific is in an El Nino or La Nina state? What are the differences in the
atmosphere and ocean for these two states?}
El Nino is a warm surface current in equatorial eastern Pacific that occurs
periodically around Christmas time.

La Nina has a stronger Walker Circulation and stronger trade winds -- more
upwelling, shallow thermocline in the east, and colder water in the west.

\subsubsection{Where in the world do deep water form?}

%&/()%&/()%&/()%&/()%&/()%&/()%&/()%&/()%&/()%&/()%&/()%&/()%&/()%&/()%&/()%&/()
