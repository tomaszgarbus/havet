%&/()%&/()%&/()%&/()%&/()%&/()%&/()%&/()%&/()%&/()%&/()%&/()%&/()%&/()%&/()%&/()
\section{Lecture 5.1: Air-Sea interaction}

\begin{itemize}
    \item Variations in solar radiation on Earth
    \item Physical properties of the atmosphere
    \item Coriolis effect impact on moving objects
    \item Global atmospheric circulation patterns
    \item Weather and climate patterns in the ocean
    \item Sea ice and icebergs
    \item Energy from wind over the oceans
\end{itemize}

\subsection{Earth's seasons}

\begin{itemize}
    \item The Earth orbits around the Sun once a year
    \item The Earth's rotation axis is tilted at 23.5$\degree$
        with respect to the ecliptic
\end{itemize}

\subsection{Uneven solar heating on Earth}

\begin{itemize}
    \item Solar energy is spread over a larger area
    \item More solar energy is reflected back into space because of the low
        angle of the sunlight
    \item Solar energy passes through more atmosphere which absorbs, scatters
        and reflects sunlight
\end{itemize}

\subsection{Earth's seasons}

\begin{itemize}
    \item NH and SH are alternately tilted toward and away from the Sun
        each year
    \item In the summer hemisphere:
        \begin{itemize}
            \item the days are longer
            \item the Sun hits the Earth at a higher angle so radiation is
                concentrated over a smaller area
            \item the higher angle also means more energy is absorbed and less
                reflected
        \end{itemize}
\end{itemize}

\subsection{Heat transport}

\begin{itemize}
    \item A net heat gain is experienced in low latitudes
    \item A net heat loss is experienced in high latitudes
    \item The equator does not get continuously hotter and the poles
        continuously cooler! Why? Heat gain and loss are balanced by oceanic
        and atmospheric circulation.
\end{itemize}

\subsection{Physical properties of the atmosphere: Composition of dry air}

Nitrogen and oxygen gas comprise 99\% of the total, with several trace gases
making up the rest; the most significant trace gas is carbon dioxide, an
important greenhouse gas.

\subsection{Physical properties of the atmopshere: Temperature}

\begin{itemize}
    \item The troposphere is lowermost 12 km of the atmosphere where most
        weather occurs
    \item Temperature in the troposphere decreases with increasing height.
    \item Outgoing long wavelength radiation from the Earth's surface
        is absorbed by GHG in the troposphere.
    \item The troposphere is primarily heated from the Earth's surface,
        not from incoming sunlight.
\end{itemize}

\subsection{Physical properties of the atmopshere: Water vapour}

\begin{itemize}
    \item Air in the troposphere contains varying amounts of water. In gas
        form, water is called water vapour.
    \item The amount of water vapour in the atmosphere depends on air
        temperature.
    \item Cool air cannot hold so much water vapor, so is typically dry.
    \item Warm air can hold more water vapor, so is typically moist.
\end{itemize}

\subsection{Physical properties of the atmopshere: Pressure}

\begin{itemize}
    \item A column of cool, dense air in the troposphere causes high pressure
        at the surface, which will lead to air diverging (moving away) at the
        surface.
\end{itemize}

\subsection{Physical properties of the atmopshere: Movement}

\begin{itemize}
    \item Air is forced from high- to low-pressure regions
    \item Moving air is called wind
\end{itemize}

\subsection{Vertical cells in the atmosphere}

\begin{itemize}
    \item The 3 vertical circulation cells:
    \begin{itemize}
        \item Hadley 0 $\degree$ -- 30 $\degree$
        \item Ferrell 30 $\degree$ -- 60 $\degree$
        \item Polar 60 $\degree$ -- 90 $\degree$
    \end{itemize}
    \item Low pressure at 0 and 60 called the equatorial low and the subpolar
        low (rising air, rain)
\end{itemize}

\subsection{Wind belts of the world}

\begin{itemize}
    \item The 3 wind belts
    \begin{itemize}
        \item Easterly trade winds
        \item Prevailing westerlies
        \item Polar easterlies
    \end{itemize}
\end{itemize}

\subsection{Weather and climate patterns: Sea and land breeze}

\begin{itemize}
    \item Rocks heat up and cool down more quickly than water
    \item The air above warm rock will be warm, leading to low air pressure
    \item The air above cool water will be cool, leading to higher air pressure
    \item Air will flow at the surface from high to low pressure
    \item The situation will reverse at night because rock cools down more
        quickly
\end{itemize}

\section{Lecture 5.1: Ocean currents}

\begin{itemize}
    \item Surface currents
        \begin{itemize}
            \item Affect upper 1000 m in the ocean
            \item Driven by major wind belts of the world
        \end{itemize}
    \item Deep currents
        \begin{itemize}
            \item Affect deep water below 1000 m depth
            \item Driven by sea water density differences
            \item Larger and slower than surface currents
        \end{itemize}
\end{itemize}

\subsection{Surface currents closely follow global wind belt pattern}

\begin{itemize}
    \item Trade winds at 0$\degree$-30$\degree$ latitude
        blow surface ocean water to the west.
    \item Prevailing westerlies at 30$\degree$-60$\degree$
        latitude blow surface ocean water from west to east
\end{itemize}

\subsection{Upwelling and downwelling}

\begin{itemize}
    \item Vertical surface layer movement (some 100s m depth)
        \begin{itemize}
            \item Upwelling = movement of underlying water to the surface
                \begin{itemize}
                    \item Lifts cold, nutrient-rich water to surface
                    \item Produces high primary productivities and abundant
                        marine life
                \end{itemize}
        \end{itemize}
\end{itemize}

%&/()%&/()%&/()%&/()%&/()%&/()%&/()%&/()%&/()%&/()%&/()%&/()%&/()%&/()%&/()%&/()