\section{Lecture 8: The coasts}

\subsection{Where is the coastline?}

The location of the coastline depends on relative sea level and 
local slope. Relative sea level is the result of the competition
between global sea level change and land uplift/subsidence
(relative to the Earth's center).

\textbf{Isostatic changes} = changes in continent level

\textbf{Eustatic changes} = changes in sea level

\subsection{Coastal waters -- shelf areas}

The width of the shelf varies a lot. The shelves generally
wide in passive margins, and narrow near deep sea trenches
(subduction zones).

\subsection{Passive margins -- coasts without a plate boundary}

\subsection{Passive margin (e.g. Atlantic coasts)}

\begin{itemize}
    \item Generally wide shelves where sediments accumulate
    \item Generally low relief (quite flat)
\end{itemize}

\subsection{Active margin -- subduction zone}

\begin{itemize}
    \item Generally narrow shelves
    \item Generally steep topography
\end{itemize}

\subsection{Coastal landforms and terminology}

\begin{itemize}
    \item berm = standterass
    \item bar = (sand-)revel
    \item longshore = kustparallell
    \item notch = strandhak
    \item wave-cut cliff = klint
    \item trough = ränna/tråg
    \item wave-cut bench = strandflate
\end{itemize}

\subsection{Sea floors are quite flat}

\subsection{Movement of sand in/out of the beach}

Note that movement both in/out and along the beach occurs.
Treating them separately makes it easier to understand.

Movement perpendicular to shorelines (in/out):
\begin{itemize}
    \item caused by breaking waves
    \item weak wave activity moves sand up the beach face toward the berm
    \item strong wave activity moves sand out from the shore to the longshore bars
\end{itemize}

\subsection{Movement of sand and water along the beach}

\subsection{Longshore current $\rightarrow$ longshore drift}

\textbf{Longshore current}

\begin{itemize}
    \item Zig-zag movement of water in the surf zone
    \item Speed increases with increasing beach slope, angle between beach and
        waves, wave height and wave frequency
\end{itemize}

\subsection{Rip currents}

\begin{itemize}
    \item Narrow, fast, outward surface currents between breaking swells
    \item Note, Do NOT swim towards the beach. To escape a rip current, swim
        sideways along the beach until clear of the current.
\end{itemize}

\subsection{Features of erosional shores -- rising sea level}

Deposition and erosion is common on all shorelines. But some shorelines
are dominated by erosion or by deposition $\rightarrow$ indicates relative
sea level.

\begin{itemize}
    \item Erosional shores are typical of areas where tectonic uplift
        is much slower than sea level rise.
\end{itemize}

\subsection{Delta}

\begin{itemize}
    \item Some rivers carry more sediment to the shoreline than can be carried
        away by longshore currents.
    \item When sediment deposition is larger than coastal erosion,
        a \textbf{bird-foot delta} is formed.
    \item When tides are the dominating process, the delta forms several
        funnel-shaped river mouths, because the rising and sinking sea level
        prevents a single channel to dominate.
\end{itemize}

\subsection{Global sea level -- eustasy}

Rapid Eustatic changes are caused by variations in ocean water
volume through ice sheets on the continents or thermal expansion
(warm water is larger than cold water).

\subsection{Evidence of emerging and submerging shorelines}

\textbf{Emergent (exposed) features}: Marine terraces, raised beaches

\textbf{Submergent (drowned) features}: Drowned beaches, drowned river valleys

\subsection{The formation of a wave cut platform/bench}

\subsection{Interfering with sand movement}

Building hard structures on the beach into the sea changes
the longshore current and longshore drift directions

Result: different distribution of sediment on the coastline

Sediments are deposited upstream of a groin and are eroded
downstream of a groin.

\subsection{Breakwaters -- shelter for boats ...and sand}

\begin{itemize}
    \item Deposition in harbor and erosion downstream
    \item Sand must be dredged regularly
\end{itemize}

\section{Lecture 8: Coastal waters}

\subsection{Salinity differences in the shallow coastal ocean}

\subsection{Temperature in the shallow coastal ocean}

Strong thermoclines mainly in mid-latitudes, seasonal

\subsection{Estuaries}

\begin{itemize}
    \item Partially enclosed coastal bodies of water where
        fresh runoff mixes with salty ocean water.
    \item Large variations in salinity and temperature.
\end{itemize}

\subsection{Coastal wetlands}

\begin{itemize}
    \item Coastal wetlands = water saturated land areas that border
        coastal environments
    \item Periodically submerged by ocean water
    \item Oxygen poor sediment $\rightarrow$ slow/no degradation
        $\rightarrow$ accumulate organic-rich peat deposits
    \item Two most important types of coastal wetlands:
        \begin{itemize}
            \item Salt marshes (mid-latitudes)
            \item Mangrove swamps (low-latitudes)
        \end{itemize}
\end{itemize}