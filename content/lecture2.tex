\subsection{Lecture 2: Chemistry of the oceans, the basics}

\subsection{Red Shift}

1929 Edwin Hubble discovered that the distance to galaxies and
their velocity were related.

\subsection{Big Bang Nucleosynthesis}

Alpher, Bethe and Gamov

\begin{itemize}
    \item Made first calculations of the origin of elements in late 1940s
    \item Initially protons, neutrons and electrons. After a short time,
        they were combined to hydrogen, helium and small amount of
        Lithium and $^7$Be through fusion
    \item Other elements need stars and supernovas
\end{itemize}

\subsection{Where do the elements come from}

\begin{itemize}
    \item In 1920 Arthur Eddington suggested that nuclear fusion is the fuel
        in stars and possibly the heavier elements are formed in stars
    \item It was not until 1957 that there was an understanding where the
        elements come from when a paper by Burbidge et al was published. The
        man behind the paper was Fred Hoyle.
\end{itemize}

\subsection{Elemental abundance in the solar system}

Fusion in stars, exploding stars are the source for the elements. S and R process
need free neutrons in order to form the heaviest elements.

\subsection{What is the Sun made of, a star close to us}

\begin{itemize}
    \item Until the 1920s the Sun was thought to be a very hot rock
        consisting of mostly iron
    \item Cecilia Payne showed in 1925 that the Sun was mainly composed of
        hydrogen and helium. Nobody believed her at first, but this was later
        accepted in 1929 and shown in another way by the person that did not
        believe her. All stars are mainly composed of hydrogen and helium, where
        hydrogen is fused to helium releasing energy.
\end{itemize}

\subsection{Formation of the solar system and Earth}

\begin{itemize}
    \item Earth was formed 4.6B years ago from a cloud of gas and dust (a nebula) where
        the elements came from stars (and supernovas) in our part of space
    \item Planet properties reflect their proximity to the early sun
    \item Some planets have experienced major perturbations and/or collisions (Venus, Earth)
    \item Comets and asteroids are debris left over from solar system formation
\end{itemize}

\subsection{How water came to Earth}

Earth was from start a hot rock, which means that condensation of water at formation is not
possible (the Earth was too hot).

Theories about water on Earth:

\begin{enumerate}
    \item The proto-Sun emitted enormous amounts of water to space that existed as clouds around
        the sun and was incorporated into asteroids and comets
    \item Another theory is that water comes from minerals and the collision with Theia that
        created our moon
\end{enumerate}

\subsection{Composition of ocean/atmosphere}

The chemical composition of oceans and atmosphere can be seen as a result of a reaction:

Minerals + volcanic gases = atmosphere + sea water + sediment

\subsection{Thermohaline circulation (thermo = temperature, haline = salt)}

Driven by global differences in density gradients, where colder and saltier waters
have higher densities than warmer and less salty waters.

\subsection{Lysocline and CCD}

Lysocline = where the rate of dissolution starts to increase rapidly

CCD = Carbon Compensation Depth, where the rate of dissolution equals production

Solubility of carbonates depends on:

\begin{itemize}
    \item Ion concentrations
    \item Saturation degree
    \item Pressure: solubility increase with pressure
    \item Temperature: solubility increase with decreasing temperature
    \item CO$_2$ pressure: solubility increase with increasing CO$_2$
\end{itemize}