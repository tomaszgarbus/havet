\section{Chapter 4}

\subsection{Neritic and pelagic deposits}

Marine sedimentary deposits can be categorized as either neritic or pelagic.

Neritic deposits (\textit{neritos} -- of the coast) are found on continental
shelves and in shallow water near islands; these deposits are generally coarse
grained. Pelagic deposits (\textit{pelagios} -- of the sea)  are found in the
deep ocean basins and are typically fine grained. 

\subsection{Types of sediment}

\subsubsection{Lithogenous}
\begin{itemize}
	\item originates from the weathering of rocks
	\item the greatest quantity of lithogenous material is found around
		the margins of the continents, where it is constantly moved by
		high-energy currents along the shoreline and in deeper
		turbidity currents
	\item the majority of lithogenous deposits -- such as  beach sands --
		are composed primarily of quartz
	\item lithogenous sediment dominates most neritic deposits
	\item beach deposits -- beach materials are composed mostly of
		quartz-rich sand that is washed down to the coast by rivers
		but can also be composed of wide variety of sizes and
		compositions
	\item continental shelf deposits -- at the end of the last ice age
		(10ka) glaciers melted and sea level rose; currently rivers
		typically drop sediment in their drowned river mouths, but in
		geologic past it would travel further to the continental shelf
	\item turbidite deposits -- turbidity currents are underwater
		avalanches that periodically move down the continental slopes
		and carve submarine canyons; they also carry a vast number of
		neritic material; this material spreads out as deep-sea fans,
		comprises the continental rise, and gradually thins toward the
		abyssal plains
	\item glacial deposits -- poorly sorted deposits containing particles
		ranging from boulders to clays, found in high-latitude portions
		of the continental shelf
	\item most pelagic deposits are composed of fine-grained material
		that accumulates slowly on the deep-ocean floor; pelagic
		lithogenous sediment includes particles that have come from
		volcanic eruptions, windblown dust, and fine material that is
		carried by deep-ocean currents
	\item abyssal clay -- composed of at least 70\% (by weight) fine,
		clay-sized particles from the continents; because they contain
		oxidized iron, they are commonly red-brown or buff in color
		and sometimes referred to as red clays
\end{itemize}

\subsubsection{Biogenous}
\begin{itemize}
	\item from hard parts of living organisms (shells, bones, and teeth)
	\item can be classified as either macroscopic or microscopic
	\item macroscopic biogenous sediment is large enough to be seen
		without a microscope and includes shells, bones, and teeth of
		large organisms; relatively rare
	\item microscopic biogenous sediment contains particles so small they
		can only be seen with a microscope; microscopic organisms
		produce tiny shells called tests (\textit{testa} -- shell) that
		begin to sink after the organisms die and continually rain down
		in great numbers onto the ocean floor
	\item the microscopic tests accumulate on the deep-ocean floor and form
		deposits called ooze (\textit{wose} -- juice)
	\item biogenous sediment is mostly created by algae and protozoans
	\item the two most common chemical compounds in biogenous sediment are
		calcium carbonate (CaCO$_3$, which forms calcite) and silica
		(SiO$_2$)
	\item two significant sources of calcium carbonate biogenous ooze are
		the foraminifers (\textit{foramen} -- an opening) and
		microscopic algae called coccolithophores
	\item when the organism dies, the individual plates (called coccoliths)
		disaggregate and can accumulate on the ocean floor as
		coccolith-rich ooze; this ooze then lithifies as chalk
	\item the White Cliffs of southern England are composed of hardened
		coccolith-rich calcium carbonate ooze, which was deposited on
		the ocean floor and has been uplifted onto land
	\item biogenous sediment is one of the most common types of pelagic
		deposits
	\item biogenous carbonate deposits are common in some areas
	\item most limestones contain fossil marine shells, suggesting a
		biogenous origin, while other carbonate-containing rocks appear
		to have been formed directly from seawater without the help of
		any marine organism
	\item ancient marine carbonate deposits constitute 2\% of Earth's
		crust and 25\% of all sedimentary rocks on Earth
	\item stromatolites -- cyanobacteria produce these deposits by trapping
		fine sediment in mucous mats; other types of algae produce long
		filaments that bind carbonate particles together
	\item microscopic biogenous sediment (ooze) is common on the
		deep-ocean floor because there is so little lithogenous
		sediment deposited at great distances from the continents that
		could dilute the biogenous material
	\item calcareous ooze can deposit on top of the mid-ocean ridge and
		then as it spreads and goes below CCD it is already covered
		and protected by abyssal clay
\end{itemize}

\subsubsection{Hydrogenous}
\begin{itemize}
	\item chemical reactions within seawater cause certain minerals to
		come out of solution (to precipitate)
	\item precipitation usually occurs when there is a change in conditions
		such as in temperature or pressure
	\item manganese nodules are rounded, hard lumps of manganese, iron and
		other metals typically 5 centimeters in diameter up to a
		maximum of about 20 centimeters
	\item when cut in half, they often reveal a layered structure formed
		by precipitation around a central nucleation object
	\item the formation of manganese nodules requires extremely low
		rates of lithogenous or biogenous input so that the nodules
		are not buried
	\item 
\end{itemize}

\subsubsection{Cosmogenous}

\subsection{Glossary of terms}

\begin{itemize}
	\item \textbf{sediments} -- 
	\item \textbf{suspension settling} -- 
	\item \textbf{texture} -- 
	\item \textbf{core} -- 
	\item \textbf{rotary drilling} -- 
	\item \textbf{paleoceanography} -- the study of how the ocean,
		atmosphere, and land have interacted in the past to produce
		changes in ocean chemistry, circulation, biology, and climate
	\item \textbf{lithogenous sediment} -- 
	\item \textbf{terrigenous sediment} -- 
	\item \textbf{biogenous sediment} -- 
	\item \textbf{hydrogenous sediment} -- 
	\item \textbf{cosmogenous sediment} -- 
	\item \textbf{quartz} -- 
	\item \textbf{grain size} -- 
	\item \textbf{sorting} -- 
	\item \textbf{neritic deposits} -- 
	\item \textbf{pelagic deposits} -- 
	\item \textbf{turbidity currents} -- 
	\item \textbf{ice rafting} -- 
	\item \textbf{abyssal clay} -- 
	\item \textbf{tests} -- 
	\item \textbf{ooze} -- 
	\item \textbf{protozoans} -- 
	\item \textbf{calcium carbonate} -- 
	\item \textbf{silica} -- 
	\item \textbf{diatom} -- 
	\item \textbf{radiolarian} -- 
	\item \textbf{plankton} -- 
	\item \textbf{diatomaceous earth} -- 
	\item \textbf{foraminifers} -- 
	\item \textbf{coccolithophores} -- 
	\item \textbf{nannoplankton} -- 
	\item \textbf{chalk} -- 
	\item \textbf{calcareous ooze} -- 
	\item \textbf{limestone} -- 
	\item \textbf{coccolith} -- 
	\item \textbf{carbonate} -- 
	\item \textbf{stromatolites} -- 
	\item \textbf{lysocline} -- 
	\item \textbf{calcite compensation depth (CCD)} -- 
	\item \textbf{upwelling} -- 
	\item \textbf{precipitate} -- 
	\item \textbf{maganese nodules} -- 
	\item \textbf{phosphates} -- 
	\item \textbf{aragonite} -- 
	\item \textbf{oolites} -- 
	\item \textbf{metal sulfides} -- 
	\item \textbf{evaporite materials} -- 
	\item \textbf{spherules} -- 
	\item \textbf{meteor} -- 
	\item \textbf{tektiles} -- 
	\item \textbf{meteorite} -- 
	\item \textbf{petroleum} -- 
	\item \textbf{gas hydrates} -- 
	\item \textbf{methane hydrates} -- 
	\item \textbf{salt deposits} -- 
	\item \textbf{phosphorite} -- 
	\item \textbf{crusts} -- 
\end{itemize}
