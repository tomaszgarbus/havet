\section{Lecture 11: The oceans and climate change}

\begin{itemize}
    \item The greenhouse effect: a look at Venus and Mars
    \item What is the role of the ocean for Earth's climate
    \item Earth's climate in the past: ice ages
    \item The ocean and ongoing and future climate change
\end{itemize}

\subsection{The oceans and climate change}

Solar radiation heats the Earth's surface directly. However, the
surface is mainly cooled by evaporation. Greenhouse gases and
clouds in the atmosphere radiate back energy to space.

\subsection{The Greenhouse effect}

Arrhenius (1896) tried to calculate the warming (cooling) of
Earth's surface temperature caused by increased (decreased)
CO$_2$. His result suggested that a CO$_2$-doubling causes
a 4$\degree$C increase.

IPCC AR6 (2021) estimates that a CO$_2$-doubling causes a warming
of $2.5$-$5\degree$C increase. This is based on a combination of
models and observations.

CO$_2$ ppm is seasonal because leaves take up CO$_2$ in the summer
and return it to the air when they fall and degrade.

\subsection{The greenhouse effect}

\begin{itemize}
    \item The atmosphere is relatively transparent to solar
        radiation, which heats the surface.
    \item The atmosphere absorbs infrared radiation emitted by the
        surface, which is re-emitted both downward and upwards.
    \item This makes the Earth's surface temperature higher than it
        would have been in the absence of atmospheric absorption
        of infrared radiation.
\end{itemize}

\subsection{Key features for planetary climates}

\begin{itemize}
    \item Solar constant
    \item Planetary albedo; fraction of reflected sunlight
    \item Black body radiation (Stefan-Boltzmann law)
    \item Emission temperature (black body)
\end{itemize}

\begin{tabular}{|c|c|c|c|}
     \hline
     & Surface T & Emission T & Greenhouse  \\
     \hline
     Venus & +460 & -21 & + 481 \\
     \hline
     Earth & +15 & -18 & +33 \\
     \hline
     Mars & -53 & -56 & +3 \\
     \hline
\end{tabular}

\begin{tabular}{|c|c|c|c|}
     \hline
     & Albedo & Surface pressure & Atmosphere  \\
     \hline
     Venus & 0.59 & 90 & CO$_2$ \\
     \hline
     Earth & 0.3 & 1 & N$_2$, O$_2$ \\
     \hline
     Mars & 0.16 & 0.01 & CO$_2$ \\
     \hline
\end{tabular}

\subsection{Maritime and continental climates}

\begin{itemize}
    \item Maritime (coastal) climate: mild winters and cool summers;
        the large ocean heat capacity reduces the summer to winter
        temperature difference.
    \item Continental (inland) climate: cold winters and hot summers.
\end{itemize}

\subsection{Net evaporation and surface salinity}

The surface salinity mirrors roughly the net evaporation,
which is defined as the evaporation minus precipitation.
High salinities in the North Atlantic promotes formation of deep
waters there.

\subsection{Causes of climate change and variability}

\begin{itemize}
    \item Internal variability (e.g. El Nino-La Nina)
    \item Vulcanism (ejects dust and CO$_2$)
    \item Variations in Earth's orbit; glacial cycles
    \item Plate tectonics; ocean-land distribution and mountains
    \item Changes in the atmospheric composition
\end{itemize}

\subsection{The oceans and climate change}

Eccentricity: ca. 100 000 years

Obliquity: ca. 40 000 years

Precession: ca. 20 000 years

\subsection{Observed sea level change}

Since 1990 roughly half of the sea level increase is due to
thermal expansion and the other half due to melting glaciers.
Currently, melting from Greenland gives 25\% of the sea level
increase. Only ice on land contribute to sea level increase
(not ice floating in the ocean).