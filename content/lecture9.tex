\section{Lecture 9: Waves}

\subsection{How are waves formed?}

\begin{itemize}
    \item Waves are created by energy release (disturbances) including:
    \begin{itemize}
        \item Wind
        \item Movement of media (air or fluids) of different densities
        \begin{itemize}
            \item Air-water = ocean waves
            \item Air-air = atmospheric waves
            \item Water-water = internal waves
        \end{itemize}
        \item Mass movement into the ocean (splash waves, but also
            commonly grouped into the category of tsunamis)
        \item Underwater seafloor movement (tsunami)
        \item Pull of the Moon and Sun (tides)
        \item Atmospheric pressure changes (seiches)
        \item Human activities (explosions)
    \end{itemize}
    \item All are examples of progressive waves, they move energy,
        not mass!
\end{itemize}

\subsection{Waves}

Generated by wind (air-water)

When and how they break depends on the depth and shape of the seafloor

\subsection{Tsunami}

Seafloor movement, underwater volcanism and slides.

\subsection{Internal waves}

An internal wave moving along the density interface (pycnocline)
below where surface waves occur.

\subsection{Most ocean waves are wind-generated}

\subsection{Types of progressive waves}

(progressive waves oscillate uniformly and travel without breaking)

\begin{enumerate}
    \item Longitudinal (compressional): back-and-forth particle motion
        in solid, liquid and gas
    \item Transverse (shear): side-to-side particle motion in solids only
        (not in ocean)
    \item Orbital (e.g. typical wind waves): combination of longitudinal and
        transverse in liquids and/or gas
\end{enumerate}

\subsection{Wave characteristics -- orbital motion in waves}

\begin{itemize}
    \item Orbital size decreases with depth to zero at the wave base
    \item Depth of wave base = one half wavelength (L/2), measured from
        still water level
\end{itemize}

\subsection{Wind waves and how they break when reaching the coasts}

\subsection{Waves undergo physical changes in the surf zone}

\subsection{Wave interference patterns}

\begin{itemize}
    \item Constructive interference: troughs and crests aligned
        between two waves; increases wave height
    \item Destructive interference: troughs of one wave lined up
        with crests of another wave; decreases wave height
    \item Mixed: variable pattern
\end{itemize}

\subsection{Tsunami}

\begin{itemize}
    \item Tsunami terminology
        \begin{itemize}
            \item Often called "tidal waves" but have nothing to
                do with the tides
            \item Japanese term meaning "harbor waves"
            \item Also called "seismic sea waves"
        \end{itemize}
    \item Created by movement of the ocean floor by
        \begin{itemize}
            \item Underwater fault movement
            \item Underwater slides
            \item Underwater volcanic eruptions
        \end{itemize}
\end{itemize}

Tsunami waves have much much longer wavelength than wind waves,
but much much lower amplitude.

\subsection{Coastal effects of tsunami}

\begin{itemize}
    \item If a trough arrives first, it appears as a strong
        withdrawal of water (similar to an extreme and
        suddenly-occuring low tide)
    \item If a crest arrives first, it appears as a strong surge
        of water that can raise sea level many meters and flood
        inland areas
    \item Tsunami often occurs as a series of surges and
        withdrawals over hours
\end{itemize}

\section{Lecture 9: Tides}

What are tides?
\begin{itemize}
    \item Tides are the daily periodic raising and lowering of sea level
    \item Tides are very long and regular shallow water waves
\end{itemize}

What causes tides?
\begin{itemize}
    \item Gravity
    \begin{itemize}
        \item Gravitational force of the Moon and Sun on Earth
            \begin{itemize}
                \item If mass increases then gravitational force increases
                \item If distance increases, then gravitational force
                    greatly decreases
            \end{itemize}
        \item Centripetal (center-seeking) gravity force required to keep
            planets/moon/sun in nearly circular orbits
    \end{itemize}
\end{itemize}

\subsection{Effect of elliptical orbits of Sun and Moon}

\begin{itemize}
    \item Tidal ranges are greater when
        \begin{itemize}
            \item The Moon is at perigee
            \item The Earth is at perihelion
        \end{itemize}
\end{itemize}

\subsection{Tidal patterns}

\begin{itemize}
    \item Diurnal: one high and one low tide each (lunar) day
    \item Semidiurnal: two high and two low tides of about the
        same height daily
    \item Mixed: characteristics of both diurnal and semidiurnal with
        successive high and/or low tides having significantly
        different heights
\end{itemize}

\subsection{Summary of tides on an idealized Earth}

\begin{itemize}
    \item All locations except the poles have two high tides
        and two low tides per lunar day
\end{itemize}

